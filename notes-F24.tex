\documentclass[12pt]{tufte-book}
\usepackage{amsthm,amssymb,amsmath,thmtools,datetime,tikz}
\setcounter{secnumdepth}{3}

\declaretheorem[numberwithin=chapter,shaded={bgcolor=Lavender}]{definition}

\declaretheorem[numberwithin=chapter,shaded={bgcolor=Thistle}]{lemma}
\declaretheorem[numberwithin=chapter,shaded={bgcolor=Thistle}]{claim}

\declaretheorem[numberwithin=chapter,shaded={bgcolor=Apricot}]{theorem}

\declaretheorem[numberwithin=chapter,shaded={bgcolor=yellow}]{remark}
\declaretheorem[numberwithin=chapter]{exercise}
\declaretheorem[numberwithin=chapter,shaded={bgcolor=pink}]{construction}
\usepackage[
    type={CC},
    modifier={by-nc-nd},
    version={4.0},
]{doclicense}

\usepackage{graphicx,xcolor,mdframed}
%\usepackage[version=0.96]{pgf}
\usepackage{enumitem}


\def\chpcolor{blue!45}
\def\chpcolortxt{blue!60}

\iffalse
\titleformat{\chapter}%
  {\huge\rmfamily\itshape\color{red}}% format applied to label+text
  {\llap{\colorbox{red}{\parbox{1.5cm}{\hfill\itshape\huge\color{white}\thechapter}}}}% label
  {2pt}% horizontal separation between label and title body
  {}% before the title body
  []% after the title body
\fi

\hypersetup{colorlinks}% uncomment this line if you prefer colored hyperlinks (e.g., for onscreen viewing)

%%
% Book metadata
\title{A Course in Theory of Cryptography}
\author[Sanjam Garg]{Sanjam Garg}
%\publisher{Publisher of This Book}

%%
% If they're installed, use Bergamo and Chantilly from www.fontsite.com.
% They're clones of Bembo and Gill Sans, respectively.
%\IfFileExists{bergamo.sty}{\usepackage[osf]{bergamo}}{}% Bembo
%\IfFileExists{chantill.sty}{\usepackage{chantill}}{}% Gill Sans

%\usepackage{microtype}

%%
% Just some sample text
\usepackage{lipsum}
\input{macros}
%%
% For nicely typeset tabular material
\usepackage{booktabs}

\usepackage[n,advantage,operators,sets,adversary,landau,probability,notions,logic,ff,mm,primitives,events,complexity,oracles,asymptotics,keys]{cryptocode} 
%%
% For graphics / images
\usepackage{graphicx,algpseudocode}
\setkeys{Gin}{width=\linewidth,totalheight=\textheight,keepaspectratio}
\graphicspath{{graphics/}}

% The fancyvrb package lets us customize the formatting of verbatim
% environments.  We use a slightly smaller font.
\usepackage{fancyvrb}
\fvset{fontsize=\normalsize}

%%
% Prints argument within hanging parentheses (i.e., parentheses that take
% up no horizontal space).  Useful in tabular environments.
\newcommand{\hangp}[1]{\makebox[0pt][r]{(}#1\makebox[0pt][l]{)}}

%%
% Prints an asterisk that takes up no horizontal space.
% Useful in tabular environments.
\newcommand{\hangstar}{\makebox[0pt][l]{*}}

%%
% Prints a trailing space in a smart way.
\usepackage{xspace}

%%
% Some shortcuts for Tufte's book titles.  The lowercase commands will
% produce the initials of the book title in italics.  The all-caps commands
% will print out the full title of the book in italics.
\newcommand{\vdqi}{\textit{VDQI}\xspace}
\newcommand{\ei}{\textit{EI}\xspace}
\newcommand{\ve}{\textit{VE}\xspace}
\newcommand{\be}{\textit{BE}\xspace}
\newcommand{\VDQI}{\textit{The Visual Display of Quantitative Information}\xspace}
\newcommand{\EI}{\textit{Envisioning Information}\xspace}
\newcommand{\VE}{\textit{Visual Explanations}\xspace}
\newcommand{\BE}{\textit{Beautiful Evidence}\xspace}

\newcommand{\TL}{Tufte-\LaTeX\xspace}

% Prints the month name (e.g., January) and the year (e.g., 2008)
\newcommand{\monthyear}{%
  \ifcase\month\or January\or February\or March\or April\or May\or June\or
  July\or August\or September\or October\or November\or
  December\fi\space\number\year
}


% Prints an epigraph and speaker in sans serif, all-caps type.
\newcommand{\openepigraph}[2]{%
  %\sffamily\fontsize{14}{16}\selectfont
  \begin{fullwidth}
  \sffamily\large
  \begin{doublespace}
  \noindent\allcaps{#1}\\% epigraph
  \noindent\allcaps{#2}% author
  \end{doublespace}
  \end{fullwidth}
}

% Inserts a blank page
\newcommand{\blankpage}{\newpage\hbox{}\thispagestyle{empty}\newpage}

\usepackage{units}

% Typesets the font size, leading, and measure in the form of 10/12x26 pc.
\newcommand{\measure}[3]{#1/#2$\times$\unit[#3]{pc}}

% Macros for typesetting the documentation
\newcommand{\hlred}[1]{\textcolor{Maroon}{#1}}% prints in red
\newcommand{\hangleft}[1]{\makebox[0pt][r]{#1}}
\newcommand{\hairsp}{\hspace{1pt}}% hair space
\newcommand{\hquad}{\hskip0.5em\relax}% half quad space
\newcommand{\TODO}{\textcolor{red}{\bf TODO!}\xspace}
\newcommand{\ie}{\textit{i.\hairsp{}e.}\xspace}
\newcommand{\eg}{\textit{e.\hairsp{}g.}\xspace}
\newcommand{\na}{\quad--}% used in tables for N/A cells
\providecommand{\XeLaTeX}{X\lower.5ex\hbox{\kern-0.15em\reflectbox{E}}\kern-0.1em\LaTeX}
\newcommand{\tXeLaTeX}{\XeLaTeX\index{XeLaTeX@\protect\XeLaTeX}}
% \index{\texttt{\textbackslash xyz}@\hangleft{\texttt{\textbackslash}}\texttt{xyz}}
\newcommand{\tuftebs}{\symbol{'134}}% a backslash in tt type in OT1/T1
\newcommand{\doccmdnoindex}[2][]{\texttt{\tuftebs#2}}% command name -- adds backslash automatically (and doesn't add cmd to the index)
\newcommand{\doccmddef}[2][]{%
  \hlred{\texttt{\tuftebs#2}}\label{cmd:#2}%
  \ifthenelse{\isempty{#1}}%
    {% add the command to the index
      \index{#2 command@\protect\hangleft{\texttt{\tuftebs}}\texttt{#2}}% command name
    }%
    {% add the command and package to the index
      \index{#2 command@\protect\hangleft{\texttt{\tuftebs}}\texttt{#2} (\texttt{#1} package)}% command name
      \index{#1 package@\texttt{#1} package}\index{packages!#1@\texttt{#1}}% package name
    }%
}% command name -- adds backslash automatically
\newcommand{\doccmd}[2][]{%
  \texttt{\tuftebs#2}%
  \ifthenelse{\isempty{#1}}%
    {% add the command to the index
      \index{#2 command@\protect\hangleft{\texttt{\tuftebs}}\texttt{#2}}% command name
    }%
    {% add the command and package to the index
      \index{#2 command@\protect\hangleft{\texttt{\tuftebs}}\texttt{#2} (\texttt{#1} package)}% command name
      \index{#1 package@\texttt{#1} package}\index{packages!#1@\texttt{#1}}% package name
    }%
}% command name -- adds backslash automatically
\newcommand{\docopt}[1]{\ensuremath{\langle}\textrm{\textit{#1}}\ensuremath{\rangle}}% optional command argument
\newcommand{\docarg}[1]{\textrm{\textit{#1}}}% (required) command argument
\newenvironment{docspec}{\begin{quotation}\ttfamily\parskip0pt\parindent0pt\ignorespaces}{\end{quotation}}% command specification environment
\newcommand{\docenv}[1]{\texttt{#1}\index{#1 environment@\texttt{#1} environment}\index{environments!#1@\texttt{#1}}}% environment name
\newcommand{\docenvdef}[1]{\hlred{\texttt{#1}}\label{env:#1}\index{#1 environment@\texttt{#1} environment}\index{environments!#1@\texttt{#1}}}% environment name
\newcommand{\docpkg}[1]{\texttt{#1}\index{#1 package@\texttt{#1} package}\index{packages!#1@\texttt{#1}}}% package name
\newcommand{\doccls}[1]{\texttt{#1}}% document class name
\newcommand{\docclsopt}[1]{\texttt{#1}\index{#1 class option@\texttt{#1} class option}\index{class options!#1@\texttt{#1}}}% document class option name
\newcommand{\docclsoptdef}[1]{\hlred{\texttt{#1}}\label{clsopt:#1}\index{#1 class option@\texttt{#1} class option}\index{class options!#1@\texttt{#1}}}% document class option name defined
\newcommand{\docmsg}[2]{\bigskip\begin{fullwidth}\noindent\ttfamily#1\end{fullwidth}\medskip\par\noindent#2}
\newcommand{\docfilehook}[2]{\texttt{#1}\index{file hooks!#2}\index{#1@\texttt{#1}}}
\newcommand{\doccounter}[1]{\texttt{#1}\index{#1 counter@\texttt{#1} counter}}

% Generates the index
\usepackage{makeidx}
\makeindex

\begin{document}
\iffalse
% Front matter
\frontmatter

% r.1 blank page
\blankpage


% v.2 epigraphs
\newpage\thispagestyle{empty}
\openepigraph{%
The public is more familiar with bad design than good design.
It is, in effect, conditioned to prefer bad design, 
because that is what it lives with. 
The new becomes threatening, the old reassuring.
}{Paul Rand%, {\itshape Design, Form, and Chaos}
}
\vfill
\openepigraph{%
A designer knows that he has achieved perfection 
not when there is nothing left to add, 
but when there is nothing left to take away.
}{Antoine de Saint-Exup\'{e}ry}
\vfill
\openepigraph{%
\ldots the designer of a new system must not only be the implementor and the first 
large-scale user; the designer should also write the first user manual\ldots 
If I had not participated fully in all these activities, 
literally hundreds of improvements would never have been made, 
because I would never have thought of them or perceived 
why they were important.
}{Donald E. Knuth}
\fi

% r.3 full title page
\maketitle


% v.4 copyright page
%\newpage
\begin{fullwidth}
~\vfill
\thispagestyle{empty}
\setlength{\parindent}{0pt}
\setlength{\parskip}{\baselineskip}
Copyright \copyright\ \the\year\ \thanklessauthor

%\par\smallcaps{Published by \thanklesspublisher}

\par\smallcaps{This document is continually being updated. Please send us your feedback.}


\par \doclicenseThis
 \index{license}

\par\textit{This draft was compiled on \today.}
\end{fullwidth}

% r.5 contents
\tableofcontents

%\listoffigures

%\listoftables

% r.7 dedication
\iffalse
\cleardoublepage
~\vfill

\begin{doublespace}
\noindent\fontsize{18}{22}\selectfont\itshape
\nohyphenation
Dedicated to those who appreciate \LaTeX{} 
and the work of \mbox{Edward R.~Tufte} 
and \mbox{Donald E.~Knuth}.
\end{doublespace}
\vfill
\vfill

% r.9 introduction
\cleardoublepage
\fi
\chapter*{Preface}
Cryptography enables many paradoxical objects, such as public key encryption, verifiable electronic signatures, zero-knowledge protocols, and fully homomorphic encryption.  The two main steps in developing such seemingly impossible primitives are (i) defining the desired security properties formally and (ii) obtaining a construction satisfying the security property provably. In modern cryptography, the second step typically assumes (unproven) computational assumptions, which are conjectured to be computationally intractable. In this course, we will define several cryptographic primitives and argue their security based on well-studied computational hardness assumptions. However, we will largely ignore the mathematics underlying the assumed computational intractability assumptions.

\section*{Acknowledgements}
These lecture notes are based on scribe notes taken by students in CS 276 over the years. Also, thanks to Peihan Miao, Akshayaram Srinivasan, and Bhaskar Roberts for helping to improve these notes.
%%
% Start the main matter (normal chapters)
\newcommand{\sanjam}[1]{{\color{red} Sanjam: #1}}

\newcommand{\bhaskar}[1]{{\color{ForestGreen} Bhaskar: #1}}

\mainmatter
\input{lec00-F24}
\input{lec01-F24}
\input{lec02-F24}
\input{lec03-F24}
\input{lec04-F24}
\input{lec05-F24}
\input{lec06-F24}
\input{lec07-F24}
\input{lec08-F24}
\input{lec09-F24}
\input{lec10-F24}
\input{lec11-F24}
\input{lec12-F24}
\input{lec13-F24}
\section{Cramer-Shoup Construction}

The Cramer-Shoup cryptosystem is a public-key encryption scheme that achieves security against adaptive chosen ciphertext attacks (CCA2). To understand its significance, let's first examine the ElGamal encryption scheme and its limitations.

\subsection{Background: ElGamal Encryption}

ElGamal encryption is a simple and elegant public-key system based on the Diffie-Hellman key exchange. The scheme operates as follows:

\begin{itemize}
    \item $\textsc{Gen}(1^n)$:
        \begin{algorithmic}
            \State Select prime $q$ where $|q|=n$
            \State Choose generator $g$ of group $G$ of order $q$
            \State Sample $x \gets \mathbb{Z}_q$ randomly
            \State Compute $h = g^x$
            \State Output $(\mathrm{pk}=(g,h), \mathrm{sk}=x)$
        \end{algorithmic}
    
    \item $\textsc{Enc}(\mathrm{pk}, m \in G)$:
        \begin{algorithmic}
            \State Sample $r \gets \mathbb{Z}_q$ randomly
            \State Compute $c_1 = g^r$
            \State Compute $c_2 = m \cdot h^r$
            \State Output $(c_1, c_2)$
        \end{algorithmic}

    \item $\textsc{Dec}(\mathrm{sk}, (c_1, c_2))$:
        \begin{algorithmic}
            \State Compute $m = c_2/c_1^x$
            \State Output $m$
        \end{algorithmic}
\end{itemize}

\textbf{Proof of Correctness:}
\begin{align*}
    c_2/c_1^x &= (m \cdot h^r)/(g^r)^x \\
    &= (m \cdot (g^x)^r)/(g^r)^x \\
    &= m \cdot g^{xr}/g^{rx} \\
    &= m
\end{align*}

However, ElGamal is malleable, meaning an adversary can modify a ciphertext to create a related ciphertext. Given a ciphertext $(c_1, c_2)$ encrypting message $m$, an adversary can create $(c_1, k \cdot c_2)$ which will decrypt to $k \cdot m$ for any $k$. This malleability makes ElGamal insecure against chosen-ciphertext attacks.

\subsection{Hash Proof Systems}

To address these limitations, we introduce Hash Proof Systems (HPS), a powerful tool for constructing CCA-secure encryption schemes.

\subsection{Formal Definition}

Let $\mathbb{X}$ be a set and $\mathbb{L} \subset \mathbb{X}$ be a language defined by:
\[ \mathbb{L} = \{x \in \mathbb{X} \mid \exists w \text{ s.t. } (x,w) \in \mathbb{R}\} \]
where $\mathbb{R}$ is a binary relation and $w$ is called a witness.

A Hash Proof System consists of three algorithms:
\begin{itemize}
    \item $\mathcal{K}\mathcal{G}\text{en}(1^n)$: Generates key pair $(pk, sk)$
    \item $\mathcal{H}_{sk}: \mathbb{X} \rightarrow \Pi$ (private evaluation)
    \item $\mathcal{H}_{pk}: \mathbb{L} \times \mathbb{W} \rightarrow \Pi$ (public evaluation)
\end{itemize}

\subsection{Key Properties}

\begin{enumerate}
    \item \textbf{Correctness:} For all $(x,w) \in \mathbb{R}$:
    \[ \mathcal{H}_{sk}(x) = \mathcal{H}_{pk}(x,w) \]
    
    \textbf{Proof:}
    This follows from the construction where both evaluations compute the same mathematical operation, but $\mathcal{H}_{pk}$ requires a witness while $\mathcal{H}_{sk}$ uses the secret key.

    \item \textbf{Smoothness:} For all $x \not\in \mathbb{L}$:
    \[ \{pk, \mathcal{H}_{sk}(x)\} \stackrel{s}{\approx} \{pk, U_\Pi\} \]
    where $U_\Pi$ is uniform over $\Pi$.
    
    \textbf{Proof Sketch:}
    This property follows from the DDH assumption in our concrete construction. When $x \not\in \mathbb{L}$, the hash value appears random to any computationally bounded adversary.

    \item \textbf{Universal Property:} For all $x \in \mathbb{L}$ and $y_1, \ldots y_t \in \mathbb{L}$:
    \[ \{pk, \mathcal{H}_{sk}(x), \{\mathcal{H}_{pk}(y_i)\}_i\} \stackrel{s}{\approx} \{pk, U_\Pi, \{\mathcal{H}_{sk}(y_i)\}_i\} \]
\end{enumerate}

\subsection{Concrete DDH-based Construction}

Let's construct a specific HPS based on the Decision Diffie-Hellman (DDH) assumption:

\begin{itemize}
    \item Let $G$ be a cyclic group of prime order $q$
    \item Fix generators $g_1, g_2 \in G$
    \item Define $\mathbb{X} = G^2$
    \item Define $\mathbb{L} = \{(h_1,h_2) \in \mathbb{X} \mid \exists r \in \mathbb{Z}_q: h_1=g_1^r \text{ and } h_2=g_2^r\}$
\end{itemize}

The construction:
\begin{align*}
    \mathcal{K}\mathcal{G}\text{en}(1^n): &\text{ Sample } d_1,d_2 \gets \mathbb{Z}_q \\
    &\text{ Output } (pk=(g_1^{d_1}g_2^{d_2}), sk=(d_1,d_2)) \\
    \mathcal{H}_{sk}(x=(h_1,h_2)) &= h_1^{d_1}h_2^{d_2} \\
    \mathcal{H}_{pk}(x=(h_1,h_2),w=r) &= pk^r
\end{align*}

\textbf{Proof of Correctness:}
For $(x,w) \in \mathbb{R}$ where $x=(g_1^r, g_2^r)$:
\begin{align*}
    \mathcal{H}_{sk}(x) &= (g_1^r)^{d_1}(g_2^r)^{d_2} \\
    &= g_1^{rd_1}g_2^{rd_2} \\
    &= (g_1^{d_1}g_2^{d_2})^r \\
    &= pk^r \\
    &= \mathcal{H}_{pk}(x,w)
\end{align*}

\subsection{Basic Scheme and Its Limitations}

Let's analyze our first attempt at constructing a secure encryption scheme using HPS:

$\mathcal{G}\text{en}(1^n)$:
\begin{itemize}
    \item $(pk,sk) \leftarrow \mathcal{K}\mathcal{G}\text{en}(1^n)$
\end{itemize}

$\text{Enc}(pk,m \in G)$:
\begin{enumerate}
    \item Sample $x \leftarrow \mathbb{L}$ along with witness $w$
    \item Compute $e = \mathcal{H}_{pk}(x,w) \cdot m$
    \item Output $(x,e)$
\end{enumerate}

$\text{Dec}(sk,(x,e))$:
\begin{itemize}
    \item Output $e/\mathcal{H}_{sk}(x)$
\end{itemize}

\subsection{Security Analysis of Basic Scheme}

While this construction achieves CPA security through the following hybrid argument:

\begin{enumerate}
    \item $\mathcal{H}_0$: Real encryption game
    \item $\mathcal{H}_1$: Replace $e = \mathcal{H}_{sk}(x) \cdot m$ (using $sk$ instead of $pk$)
        \begin{itemize}
            \item This is identical to $\mathcal{H}_0$ because $x \in \mathbb{L}$ and by HPS correctness
            \item Formally: For any $(x,w) \in \mathbb{R}$, $\mathcal{H}_{pk}(x,w) = \mathcal{H}_{sk}(x)$
        \end{itemize}
    \item $\mathcal{H}_2$: Replace $x \leftarrow \mathbb{X} \setminus \mathbb{L}$ 
        \begin{itemize}
            \item Indistinguishable by property 2 of HPS
            \item Adversary cannot tell if $x$ is sampled from $\mathbb{L}$ or $\mathbb{X} \setminus \mathbb{L}$
        \end{itemize}
    \item $\mathcal{H}_3$: Replace $e$ with uniform random element
        \begin{itemize}
            \item Indistinguishable by smoothness property of HPS
            \item When $x \not\in \mathbb{L}$, $\mathcal{H}_{sk}(x)$ is statistically close to uniform
        \end{itemize}
\end{enumerate}

However, this scheme fails to achieve CCA-2 security. Here's a concrete attack:

\begin{theorem}[CCA-2 Attack on Basic Scheme]
The basic HPS construction is not CCA-2 secure.
\end{theorem}

\begin{proof}
Consider the following attack in the CCA-2 game:
\begin{enumerate}
    \item Adversary receives challenge ciphertext $(x^*, e^*)$ for message $m_b$
    \item Adversary creates modified ciphertext $(x^*, k \cdot e^*)$ for some $k \in G$
    \item Adversary submits modified ciphertext to decryption oracle
    \item Let $m'$ be the decrypted result. Then:
    \begin{align*}
        m' &= (k \cdot e^*)/\mathcal{H}_{sk}(x^*) \\
        &= k \cdot (e^*/\mathcal{H}_{sk}(x^*)) \\
        &= k \cdot m_b
    \end{align*}
    \item Since $k$ is known, adversary can recover $m_b$ and win the CCA-2 game
\end{enumerate}
\end{proof}

\subsection{The Need for Ciphertext Integrity}

The key insight is that we need to prevent ciphertext manipulation. The full Cramer-Shoup construction addresses this by:

\begin{enumerate}
    \item Adding a second HPS instance ($\mathcal{H}'$) that acts as a "proof of well-formedness"
    \item Using a collision-resistant hash function to bind all components together
    \item Verifying the proof $\pi$ before decryption
\end{enumerate}

The second HPS instance must satisfy a stronger security property called "2-smoothness":

\begin{definition}[2-Smoothness]
For all $x_1,x_2 \in \mathbb{L}$ and $t_1,t_2 \in T$ with $t_1 \neq t_2$:
\[ \{pk, \mathcal{H}_{sk}(x_1,t_1), \mathcal{H}_{sk}(x_2,t_2)\} \stackrel{s}{\approx} \{pk, U_\Pi, U_\Pi\} \]
\end{definition}

\subsection{CCA-2 Secure Construction}

The final Cramer-Shoup construction achieves CCA-2 security by combining two HPS instances with a collision-resistant hash function:

\subsection{The Scheme}

$\mathcal{G}\text{en}(1^n)$:
\begin{algorithmic}
    \State $(pk,sk) \leftarrow \mathcal{K}\mathcal{G}\text{en}(1^n)$
    \State $(pk',sk') \leftarrow \mathcal{K}\mathcal{G}\text{en}(1^n)$
    \State Output $((pk,pk'),(sk,sk'))$
\end{algorithmic}

$\text{Enc}(pk,m)$:
\begin{algorithmic}
    \State Sample $x \leftarrow \mathbb{L}$ with witness $w$
    \State $e = \mathcal{H}_{pk}(x,w) \cdot m$
    \State $t = \text{CRHF}(x,e)$
    \State $\pi = \mathcal{H}'_{pk}(x,w,t)$
    \State Output $(x,e,\pi)$
\end{algorithmic}

$\text{Dec}(sk,(x,e,\pi))$:
\begin{algorithmic}
    \If{$\pi \neq \mathcal{H}'_{sk}(x)$}
        \State Output $\perp$
    \EndIf
    \State Output $e/\mathcal{H}_{sk}(x)$
\end{algorithmic}
\subsection{Final Security Theorem}

\begin{theorem}
If $\mathcal{H}$ is a 1-smooth HPS, $\mathcal{H}'$ is a 2-smooth HPS, and CRHF is a collision-resistant hash function, then the scheme is CCA-2 secure.
\end{theorem}

\begin{proof}
We prove security through a sequence of hybrid games, showing each transition is indistinguishable to a PPT adversary. Let $A$ be any PPT adversary in the CCA2 game.

First, we establish a crucial lemma:

\begin{lemma}[Key Soundness Lemma]
For any ciphertext decryption query $(x,e,\pi)$ that $A$ makes, if $\text{Dec}((sk,sk'),(x,e,\pi)) \neq \perp$ then $x \in \mathbb{L}$ except with negligible probability.
\end{lemma}

\begin{proof}[Proof of Lemma]
Suppose for contradiction that $x \notin \mathbb{L}$ but the decryption doesn't return $\perp$. This means:
\[ \pi = \mathcal{H}'_{sk'}(x, \text{CRHF}(x,e)) \]

By the 2-smoothness property of $\mathcal{H}'$, when $x \notin \mathbb{L}$, the value $\mathcal{H}'_{sk'}(x,t)$ is statistically indistinguishable from random, even given $pk'$. Therefore, the probability that $A$ can generate such a $\pi$ is at most $1/|\Pi|$, which is negligible.
\end{proof}

Now we proceed with the hybrid games:

\begin{itemize}
    \item $\mathcal{G}_0$: The real CCA2 game.
    
    \item $\mathcal{G}_1$: Same as $\mathcal{G}_0$, but the challenger computes the challenge ciphertext's $e^*$ component using $sk$ instead of $pk$:
    \[ e^* = \mathcal{H}_{sk}(x^*) \cdot m_b \]
    instead of
    \[ e^* = \mathcal{H}_{pk}(x^*,w) \cdot m_b \]
    
    By the correctness property of HPS, these are identical when $x^* \in \mathbb{L}$, so:
    \[ \Pr[\mathcal{G}_0 = 1] = \Pr[\mathcal{G}_1 = 1] \]

    \item $\mathcal{G}_2$: Same as $\mathcal{G}_1$, but now the challenger samples $x^* \gets \mathbb{X} \setminus \mathbb{L}$ instead of from $\mathbb{L}$.
    
    By the hardness of distinguishing elements in $\mathbb{L}$ from elements in $\mathbb{X} \setminus \mathbb{L}$ (which follows from the DDH assumption in our concrete construction):
    \[ |\Pr[\mathcal{G}_1 = 1] - \Pr[\mathcal{G}_2 = 1]| \leq \epsilon_{\text{DDH}} \]

    \item $\mathcal{G}_3$: Same as $\mathcal{G}_2$, but replace $\mathcal{H}_{sk}(x^*)$ with a uniform random value $u \gets \Pi$ in computing $e^*$:
    \[ e^* = u \cdot m_b \]
    
    By the smoothness property of $\mathcal{H}$, since $x^* \notin \mathbb{L}$:
    \[ |\Pr[\mathcal{G}_2 = 1] - \Pr[\mathcal{G}_3 = 1]| \leq \epsilon_{\text{smooth}} \]

    \item $\mathcal{G}_4$: Same as $\mathcal{G}_3$, but replace $e^*$ with a uniform random value in $G$. 
    
    Since $u$ is uniform in $\Pi$ and independent of $m_b$, $e^*$ is uniform in $G$ and independent of $m_b$, so:
    \[ \Pr[\mathcal{G}_3 = 1] = \Pr[\mathcal{G}_4 = 1] \]
\end{itemize}

In $\mathcal{G}_4$, the challenge ciphertext is independent of $m_b$, so $A$ has no advantage. Therefore:
\[ \text{Adv}^{\text{CCA2}}_A \leq \epsilon_{\text{DDH}} + \epsilon_{\text{smooth}} \]

To complete the proof, we need to show that the decryption oracle queries in all games can be answered properly. For any decryption query $(x,e,\pi)$:

\begin{enumerate}
    \item If $(x,e) = (x^*,e^*)$ but $\pi \neq \pi^*$: By 2-smoothness of $\mathcal{H}'$, generating a valid $\pi$ is infeasible.
    \item If $(x,e) \neq (x^*,e^*)$ but $\text{CRHF}(x,e) = \text{CRHF}(x^*,e^*)$: This breaks collision resistance of CRHF.
    \item If $\text{CRHF}(x,e) \neq \text{CRHF}(x^*,e^*)$: By 2-smoothness of $\mathcal{H}'$, any $\pi$ that verifies must correspond to a witness $w$ such that $(x,w) \in \mathbb{R}$, meaning $x \in \mathbb{L}$. Therefore, the decryption oracle can be simulated properly.
\end{enumerate}

Thus, all hybrid games are indistinguishable to $A$, and the scheme is CCA2-secure.
\end{proof}

\input{lec15-F24}
\input{lec16-F24}
\input{lec17-F24}
\input{lec18-F24}
\input{lec19-F24}
\input{lec20-F24}
\input{lec21-F24}
\input{lec22-F24}
\input{lec23-F24}


\section{zkSNARKs}

In this section, we will overview the fundamentals of zkSNARKs, or Zero-Knowledge Succinct Non-interactive ARguments of Knowledge.

\subsection{Preliminaries}

Before diving into zkSNARKs, let's understand the basic framework we're working with. In cryptography, we often deal with statements of the form ``I know a secret value that satisfies some property.'' For instance, we might want to prove that we know the private key corresponding to a public key, or that we know a solution to a Sudoku puzzle, or that we know a valid password for an account.

To formalize these statements, we use what's called a binary relation $R$. This relation takes two inputs: the public statement $x$ (like a Sudoku puzzle) and the secret witness $w$ (like the solution). $R$ is an efficiently computable binary relation that outputs 1 if the witness $w$ is valid for statement $x$, and 0 otherwise.

For an NP language $\mathcal{L}$, we can say that $x \in \mathcal{L}$ if and only if there exists a witness $w$ such that $R(x, w) = 1$ ($R$ being the corresponding to $\mathcal{L}$ relation). Conversely, $x \notin \mathcal{L}$ if and only if there does not exist any witness $w$ such that $R(x, w) = 1$. A crucial property is that while finding a valid witness $w$ may be computationally hard (like solving a Sudoku puzzle), verifying the relation $R(x,w)=1$ is always efficient (like checking if a Sudoku solution is valid). This verification can be done in polynomial time.

This framework allows us to express a wide variety of practical problems where we want to prove knowledge of a solution without revealing the solution itself, which is exactly what zkSNARKs will help us achieve.

\subsection{Properties of zkSNARKs}

We now introduce the properties of zkSNARKs. We seek to use zkSNARKs to prove that $x \in L \iff R(x, w) = 1$. Informally, if a prover has $(x, w)$, we must send a proof $\pi$ such that the verifier, who has the instance $x$, can efficiently check that $R(x, w) = 1$. For the purposes of this lecture, we are focusing on non-interactive proof systems. This means that there is no back-and-forth communication between the prover and the verifier, and the verifier can verify the prover's statement in one shot.

Two of the properties of zkSNARKs are common to all proof systems: Correctness ensures that if the statement is true, the prover should always be able to convince the verifier. Soundness ensures that no cheating prover can convince the verifier about an invalid statement. However, a correct and sound proof system is not particularly interesting. Indeed, we could simply achieve this by sending the witness $w$ from the prover to the verifier. We now discuss two properties that make zkSNARKs interesting: Succinctness and Zero-Knowledge. Succinctness requires that the proof $\pi$ sent by the prover is significantly smaller than the witness $w$. Specifically, the proof size $|\pi|$ must be bounded by $poly(\lambda, \log(|w|))$, where $\lambda$ is the security parameter. This means the proof grows only (at most) polylogarithmically with the witness size. Zero-knowledge ensures that the proof should not reveal any information about the witness $w$ beyond what can be deduced from the statement being proven. This property is what differentiates zkSNARKs from SNARKs.

The succinctness properties make SNARKs incredibly relevant, even for practical applications where we do not care about hiding the witness $w$. This is because the proof size is exponentially more efficient than directly sending the witness $w$ from the prover to the verifier.

\paragraph{Succinctness example.} Let's consider a practical example to illustrate the power of SNARKs' succinctness property. Suppose we have a 1 TB hard disk and want to prove to a verifier that $Hash(hard\text{ }disk) = x$ for some known hash value $x$. We have two options: Without SNARKs, we would need to send the entire 1 TB hard disk to the verifier, who then computes the hash themselves. With SNARKs, we can generate a succinct proof $\pi$ (for example 1 KB) that proves knowledge of the hard disk contents whose hash equals $x$. Even in scenarios where we do not need to hide the hard disk contents (i.e., zero-knowledge is not required), the SNARK approach is dramatically more efficient in terms of communication complexity: sending a 1 KB proof versus transferring 1 TB of data. This lecture will primarily focus on achieving succinctness, as typically adding zero-knowledge properties is a relatively straightforward extension once the basic SNARK construction is understood.

\paragraph{How to build SNARKs?} To construct SNARKs, we will follow these key steps. First, we need to find a ``SNARK-friendly'' representation model for NP languages. The example we will use is square span programs. Next, we must show that this model can capture all NP relations, using boolean circuits as an example. Then, we construct the SNARK using cryptographic techniques, specifically employing bilinear groups in our example construction. After that, we prove soundness. Finally, we add zero-knowledge properties. We will study the DFGK14 SNARK, which conceptually is in the SNARK family of the widely known Groth16 zkSNARK. DFGK14 which is conceptually simpler but employs the same fundamental cryptographic ideas.

\subsection{Square Span Programs}

We begin by introducing Square Span Programs (SSPs), which provide a ``SNARK-friendly'' representation for NP languages.

\begin{definition}[Square Span Program]
A square span program $Q$ over a field $\mathbb{F}$ consists of a size parameter $m \in \mathbb{N}$, a degree parameter $d \in \mathbb{N}$, and a set of polynomials $\{v_0(x), v_1(x), \dots, v_m(x), t(x)\}$. Each $v_i(x)$ is a polynomial over $\mathbb{F}$ of degree at most $d$, and $t(x)$ is a polynomial over $\mathbb{F}$ of degree exactly $d$.
\end{definition}

\begin{definition}[SSP Acceptance]
We say that a square span program $Q$ accepts an input $(a_1, \dots, a_\ell) \in \mathbb{F}^\ell$ if and only if there exist values $a_{\ell+1}, \dots, a_m \in \mathbb{F}$ such that $t(x)$ divides $(v_0(x) + \sum_{i=1}^m a_i v_i(x))^2 - 1$. In other words, there exists a quotient polynomial $h(x)$ such that $(v_0(x) + \sum_{i=1}^m a_i v_i(x))^2 - 1 = h(x)t(x)$.
\end{definition}

The values $a_1, \dots, a_\ell$ represent the input to our computation, while $a_{\ell+1}, \dots, a_m$ serve as auxiliary values (similar to a witness in an NP relation). As we will see, this algebraic structure is particularly well-suited for constructing SNARKs. A key property of SSPs is their expressiveness: we can transform any boolean circuit into an equivalent square span program. This transformation will be our next focus.

\subsection{From Boolean Circuits to SSPs}

Consider the boolean circuit in Figure \ref{fig:ssp}. We will transform this circuit into an SSP.

\begin{figure}[h]
  \centering
  \includegraphics[width=0.4\textwidth]{figures/ssp.pdf}
  \caption{Boolean circuit example for SSP transformation \label{fig:ssp}} 
\end{figure}

First, we express each gate type as a linear constraint. For a XOR gate, the output $a_4$ of inputs $a_1, a_2$ must satisfy $a_1 + a_2 + a_4 \in \{0, 2\}$. For an OR gate, the output $a_5$ of inputs $a_2, a_3$ must satisfy $(1-a_2) + (1-a_3) - 2(1-a_5) \in \{0, 1\}$. For a NAND gate, the output $a_6$ of inputs $a_4, a_5$ must satisfy $a_4 + a_5 - 2(1-a_6) \in \{0, 1\}$.

For the circuit to be satisfiable, all gate constraints must be satisfied, the output should be $1$ or equivalently the constraint $3(1 - a_6) \in \{0, 2\}$ must hold, and all boolean value constraints $a_i \in \{0, 1\}$ for all $i \in \{1,\ldots,6\}$ must be met. To standardize these constraints, we multiply all constraints involving $\{0, 1\}$ by 2 to normalize ranges and combine constraints into a matrix form for $a_1,\ldots,a_6$. Finally, we seek to unify the constraints into a larger matrix that encompasses all the constraints in the whole circuit, where each column represents a single boolean algebra constraint. The resulting constraint matrix $M$ is below, with columns representing the XOR, OR, NAND, and output constraints, respectively:

\[ \begin{pmatrix}
1 & 0 & 0 & 0 & 2 & 0 & 0 & 0 & 0 & 0 \\
1 & -2 & 0 & 0 & 0 & 2 & 0 & 0 & 0 & 0 \\
0 & -2 & 0 & 0 & 0 & 0 & 2 & 0 & 0 & 0 \\
1 & 0 & 2 & 0 & 0 & 0 & 0 & 2 & 0 & 0 \\
0 & 4 & 2 & 0 & 0 & 0 & 0 & 0 & 2 & 0 \\
0 & 0 & 4 & -3 & 0 & 0 & 0 & 0 & 0 & 2
\end{pmatrix} \]

There is also a constant vector $\vec{\delta}$ associated with these constraints:

\[ \vec{\delta} = [0 \; 0 \; -4 \; 3 \; | \; 0 \; 0 \; 0 \; 0 \; 0 \; 0] \]

All constraints must evaluate to elements in $\{0, 2\}^{10}$. 

\paragraph{Generalizing to Arbitrary Circuits.}

Let's now see how we can transform any arbitrary boolean circuit (with fan-in $2$, fan-out $1$ gates) into a Square Span Program. Consider a circuit with $m$ wires and $n$ gates. Our first task is to construct a matrix that captures all the constraints of our circuit. For each gate $k$ in our circuit, we create a column vector $\vec{v_k} = (v_{1k}, v_{2k}, \ldots, v_{mk})^T$ that encodes the gate's constraints. These constraints ensure the gate operates correctly—just as we saw with our XOR, OR, and NAND gates in the previous example.

After encoding all $n$ gates, we add a special column for the output constraint, which takes the form $(0,\ldots,0,-3)^T$. The $-3$ here is somewhat arbitrary; any field element different from 2 would work. We then augment our matrix with a diagonal matrix $D$ where each diagonal entry is 2. This diagonal matrix serves to enforce that all our variables are boolean, a crucial requirement for circuit satisfaction.

To complete our constraint system, we need a constant vector $\vec{\delta} = (\delta_1,\ldots,\delta_n,3,0,\ldots,0)$ where each $\delta_i$ represents the constant term for gate $i$. These constants are chosen from the set $\{0,2\}$, with the exception of the output constraint's constant which is 3.

Let's now see how we can transform any arbitrary boolean circuit into a Square Span Program. Consider a circuit with $m$ wires and $n$ gates. The complete constraint system can be written as:

\[ \begin{pmatrix} a_1 & a_2 & \cdots & a_m \end{pmatrix}
\begin{pmatrix}
v_{11} & v_{12} & \cdots & v_{1n} & 0 & 2 & 0 & \cdots & 0 \\
v_{21} & v_{22} & \cdots & v_{2n} & 0 & 0 & 2 & \cdots & 0 \\
v_{31} & v_{32} & \cdots & v_{3n} & 0 & 0 & 0 & \ddots & 0 \\
\vdots & \vdots & \ddots & \vdots & \vdots & \vdots & \vdots & \ddots & \vdots \\
v_{m1} & v_{m2} & \cdots & v_{mn} & -3 & 0 & 0 & \cdots & 2
\end{pmatrix} + \begin{pmatrix} \delta_1 & \delta_2 & \cdots & \delta_n & 3 & 0 & \cdots & 0 \end{pmatrix} \]

where $(a_1,\ldots,a_m)$ are the wire values (variables we solve for), the first $n$ columns $(v_{ij})$ represent the gate constraints, column $n+1$ is the output constraint $(0,\ldots,0,-3)^T$, the next $m$ columns form the diagonal matrix $D$ with $2$'s on the diagonal, and the constant vector $(\delta_1,\ldots,\delta_m)$ contains $\delta_i \in \{0,2\}$ for $i \leq n$ (gate constraints), $\delta_{n+1} = 3$ (output constraint), and $\delta_i = 0$ for $i > n+1$ (boolean constraints).

Next, we convert these discrete constraints into a polynomial system. We choose distinct field elements $x_1,\ldots,x_d$ (where $d=n+1+m$ is our total number of constraints) and use polynomial interpolation to create our SSP. For each row $i$ of our matrix, we construct a polynomial $v_i(x)$ that evaluates to the $(i,j)$ entry when evaluated at point $x_j$. Similarly, we create a polynomial $v_0(x)$ that interpolates our constant vector $\vec{\delta}$.

The target polynomial $t(x)$ is defined as the product of all linear terms:

\[ t(x) = \prod_{j=1}^d (x-x_j) \]

This polynomial is crucial because it ``zeros out'' exactly at our constraint points. The beauty of this construction is that it transforms our circuit satisfaction problem into an elegant polynomial divisibility question: the circuit is satisfiable if and only if there exist values $(a_1,\ldots,a_m)$ such that:

\[ \left(v_0(x) + \sum_{i=1}^m a_i v_i(x)\right)^2 - 1 = h(x)t(x) \]

for some polynomial $h(x)$.

After converting to a polynomial system, our constraint matrix becomes:

\[ \begin{pmatrix} 
a_1 \\ 
a_2 \\ 
\vdots \\ 
a_m 
\end{pmatrix}^T
\begin{pmatrix}
v_1(x_1) & v_1(x_2) & \cdots & v_1(x_n) & 0 & v_1(x_{n+2}) & 0 & \cdots & 0 \\
v_2(x_1) & v_2(x_2) & \cdots & v_2(x_n) & 0 & 0 & v_2(x_{n+2}) & \cdots & 0 \\
v_3(x_1) & v_3(x_2) & \cdots & v_3(x_n) & 0 & 0 & 0 & \ddots & 0 \\
\vdots & \vdots & \ddots & \vdots & \vdots & \vdots & \vdots & \ddots & \vdots \\
v_m(x_1) & v_m(x_2) & \cdots & v_m(x_n) & v_m(x_{n+1}) & 0 & 0 & \cdots & v_m(x_d)
\end{pmatrix} + 
\begin{pmatrix} v_0(x_1) \\ v_0(x_2) \\ \vdots \\ v_0(x_n) \\ v_0(x_{n+1}) \\ 0 \\ 0 \\ \vdots \\ 0 \end{pmatrix}^T \]

where:
\begin{itemize}
    \item Each $v_i(x_j)$ evaluates to the corresponding matrix entry at point $x_j$
    \item The diagonal entries evaluate to 2 at their respective points
    \item The output constraint column evaluates to $(0,\ldots,0,-3)$ at point $x_{n+1}$
    \item $v_0(x)$ interpolates the constant vector $(\delta_1,\ldots,\delta_n,3,0,\ldots,0)$
\end{itemize}

For any boolean circuit $C$, we can construct an SSP instance $(v_0(x),\ldots,v_m(x),t(x))$ such that $C(x_1,\ldots,x_\ell) = 1$ if and only if there exist values $(a_{\ell+1},\ldots,a_m)$ making the above polynomial division possible.

The correctness of this transformation follows from our construction: each gate's constraints are captured at distinct evaluation points, the boolean nature of our variables is enforced by the diagonal matrix, and the polynomial division condition.




Now, for notational convenience, we define $v_0'(x) = v_0(x) - 1$. 

At each evaluation point $x_j$, our circuit constraints require:
\[ \left(\sum_{i=1}^m a_i v_i(x_j) + v_0'(x_j)\right)^2 = 1 \]

This set of point-wise constraints can be unified into a single polynomial constraint:
\[ \left(\sum_{i=1}^m a_i v_i(x) + v_0'(x)\right)^2 - 1 \equiv 0 \pmod{t(x)} \]

where $t(x) = \prod_{j=1}^d (x - x_j)$ is our target polynomial.

Equivalently, there must exist some polynomial $h(x)$ such that:
\[ \left(\sum_{i=1}^m a_i v_i(x) + v_0'(x)\right)^2 - 1 = h(x)t(x) \]

This is precisely the SSP satisfiability condition. Thus, we have shown that circuit satisfiability is equivalent to the existence of coefficients $a_i$ that satisfy this polynomial divisibility condition.

\subsection{From SSPs to SNARKs}

We'll use bilinear groups to construct our SNARK. Our construction begins with a bilinear group setup:
\[ (\mathbb{F}_p, G_1, G_2, G_T, e, g_1, g_2) \]

The core idea of our construction leverages the pairing operation $e$ to verify polynomial equations in the exponent, roughly:
\[ e(g_1^{[v_0(x) + \sum_{i=1}^m a_i v_i(x)]}, g_2^{[v_0(x) + \sum_{i=1}^m a_i v_i(x)]}) = e(g_1^{h(x)}, g_2^{t(x)}) \cdot e(g_1, g_2) \]
We cannot directly argue about polynomial in the exponents; $x$ is a formal variable not a concrete value. For this we will sample a uniformly random point $\tau$ and check the polynomial equation in the exponent on $\tau$. We will keep $\tau$ secret from the prover thus, intuitively, from their point of view random.

The common reference string (CRS) forms the foundation of our construction. It consists of powers of a secret value $\tau$:
\[ \text{CRS} = (bg, g_1^\tau, g_1^{\tau^2}, \dots, g_1^{\tau^d}, g_2^\tau, g_2^{\tau^2}, \dots, g_2^{\tau^d}, g_1^{\beta v_{\ell + 1}(\tau)}, \dots, g_1^{\beta v_m(\tau)}, h_2, h_2^\beta) \]

In our protocol, we distinguish between two types of values:
\begin{itemize}
    \item Statement: $(a_1, \dots, a_\ell)$ - the public inputs
    \item Witness: $(a_{\ell+1}, \dots, a_m)$ - the private values
\end{itemize}

For proof generation, the prover computes four crucial group elements:
\begin{align*}
    \pi_1 &= g_1^{\sum_{i=\ell+1}^m a_i v_i(\tau)} \\
    \pi_2 &= g_2^{\sum_{i=\ell+1}^m a_i v_i(\tau)} \\
    \pi_3 &= g_1^{h(z)} \\
    \pi_4 &= g_1^{\beta \sum_{i=1}^m a_i v_i(\tau)}
\end{align*}

The verification process consists of two steps. First, the verifier computes intermediate values:
\begin{align*}
    \pi_1' &= g_1^{v_0(z) + \sum_{i=1}^\ell a_i v_i(\tau)} \cdot \pi_1 \\
    \pi_2' &= g_2^{v_0(z) + \sum_{i=1}^\ell a_i v_i(\tau)} \cdot \pi_2
\end{align*}

Then, the verifier performs three critical pairing checks:
\begin{align*}
    e(\pi_1', \pi_2') &= e(\pi_3, g_2^{t(\tau)}) \cdot e(g_1, g_2) \\
    e(\pi_1, g_2) &= e(g_1, \pi_2) \\
    e(\pi_1, h_2^\beta) &= e(\pi_4, h_2)
\end{align*}

These pairing equations serve distinct purposes in ensuring the proof's validity. The first check confirms that the SSP is satisfied by verifying the polynomial equation in the exponent. The second check ensures consistency between the prover's elements in groups $G_1$ and $G_2$. The final check prevents malformed group elements by verifying they were properly constructed using the CRS values.

\subsection{Soundness}

Due to the non-interactive nature of our protocol, we cannot rely on black-box extraction for our security proofs. Instead, we require a non-black-box extractor with access to the prover's code, which is acceptable within our security model.

The soundness guarantees of SNARKs, including our SSP SNARK construction, differ from traditional cryptographic protocols. They necessarily rely on non-falsifiable assumptions or the random oracle model (ROM).

Non-falsifiable assumptions represent a unique class of cryptographic assumptions. The most widely used non-falsifiable assumptions are the knowledge-assumptions, informally the simplest form of these assumptions state that given $(g, g^\alpha)$ no adversary can output a pair $(g^x, (g^x)^\alpha)$ without actually ``knowing'' the exponent $x$. The term ``non-falsifiable'' stems from the fact that these assumptions make claims about what an adversary must know, rather than just what it can or cannot compute.

Our first attempt at formalizing soundness states that for every probabilistic polynomial-time (PPT) adversary $\mathcal{A}$, there exists a PPT extractor $\mathcal{E}_\mathcal{A}$ such that:
\[ \Pr[(v_1, v_2) \gets \mathcal{A}(g, g^\alpha, z), x \gets \mathcal{E}_\mathcal{A}(g, g^\alpha, z) : v_2 = v_1^\alpha \land v_1 \neq g^x] = negl(\lambda) \]

This probability involves three key components:
\begin{itemize}
    \item $z$ represents auxiliary information available to the adversary
    \item $\lambda$ denotes the security parameter
    \item $negl(\lambda)$ represents a negligible function
\end{itemize}

Recent research has revealed limitations in this initial formulation. Specifically, certain obfuscated programs, when provided as auxiliary input $z$, enable adversaries to produce valid pairs $(v_1, v_2)$ while preventing the extractor $\mathcal{E}_\mathcal{A}$ from determining $x$ due to the complexity of reverse-engineering the obfuscated program.

\paragraph{Knowledge Soundness Definition}

To address these limitations, we've developed a more robust formulation for our SNARK construction that assumes only 'benign' auxiliary information. For all relations $R \in \mathcal{R}$ and non-uniform adversaries $\mathcal{A}$, there exists a non-uniform PPT extractor $\mathcal{E}_\mathcal{A}$ such that for all benign $z \in \mathcal{Z}$:
\[ \begin{aligned}
&\Pr[((v_1, v_2); c_0, \dots, c_d) \gets (\mathcal{A} \| \mathcal{E}_\mathcal{A})(bg, g_1, g_2, \dots, g_1^{\tau^d}, g_2^{\tau^d}, z) : \\
&\quad e(v_1, g_2) = e(g_1, v_2) \land v_1 \neq g_1^{\sum_{i=0}^d c_i \tau^i}] = negl(\lambda)
\end{aligned} \]

This refined definition captures several crucial aspects: the adversary and extractor share access to the exact same inputs and random coins, the extractor must produce coefficients $c_i$ explaining the adversary's output, the auxiliary information must be ``benign'' (excluding problematic cases like obfuscated programs), and the probability of adversarial success without the extractor finding a valid witness must be negligible.

While this non-falsifiable assumption is stronger than traditional cryptographic assumptions, it appears to be necessary for constructing efficient SNARKs using current techniques.

\paragraph{Knowledge Assumption}

We now formalize the knowledge assumption d-PKE (power knowledge of exponent).

For every non-uniform PPT adversary $\mathcal{A}$ there exists a non-uniform PPT extractor $\mathcal{E}_\mathcal{A}$ such that for every ``benign'' $z$:

$$\Pr [\mathcal{A} || \mathcal{E}_\mathcal{A} (bg, g_1^\tau, g_1^{\tau^2}, \dots, g_1^{\tau^d}, g_2^\tau, g_2^{\tau^2}, \dots, g_2^{\tau^d}, z)$$
$$\rightarrow (v_1, v_2; c_0, \dots, c_d) : e(v_1, g_2) = e(g_1, v_2) \land v_1 \neq g_1^{\sum_{i=0}^d c_i \tau^i}] = negl(\lambda)$$

This is basically saying that if the adversary can output a valid pair $(v_1, v_2)$ then the extractor can output a polynomial with coefficients $c_0, \dots, c_d$ such that $v_1 = g_1^{\sum_{i=0}^d c_i \tau^i}$. Observe this assumption is very close to what we actually construct in the SNARK.

We now show a sketch of the knowledge soundness proof.

Assuming a knowledge-sound adversary $\mathcal{A}$, we are going to construct $\mathcal{A}'$ as follows:

$$\mathcal{A}'(bg, \{g_1^{\tau^i}\}_{i=0}^d, \{g_2^{\tau^i}\}_{i=0}^d, z' = (z || \{g_1^{\beta v_i(\tau)}\}_{i=\ell+1}^m, h_1, h_2^\beta))$$

We give the rest of the CRS of the SNARK to the adversary inside of the auxiliary input $z'$. Then, this adversary runs the adversary of the knowledge soundness assumption as follows:

$$\mathcal{A}(R, z, crs) \rightarrow (\pi_1, \pi_2, \pi_3, \pi_4, (a_1, \dots, a_\ell))$$

such that the three pairing verification equation of the SNARK hold, and outputs $v_1 = \pi_1 \cdot g_1^{v_0(\tau) + \sum_{i=1}^\ell a_i v_i(\tau)}$, $v_2 = \pi_2 \cdot g_2^{v_0(\tau) + \sum_{i=1}^\ell a_i v_i(\tau)}$ where $e(v_1, g_2) = e(g_1, v_2)$. 

There is an extractor $\mathcal{E}_\mathcal{A}'$ that can output $c_0, \dots, c_d$ such that $v_1 = g_1^{\sum_{i=0}^d c_i \tau^i}$.

We now construct the extractor $\mathcal{E}_\mathcal{A}$ for knowledge soundness (using the the extractor the knowledge assumption $\mathcal{E}_{\mathcal{A}'}$). The extractor is given $v_1 = g_1^{f(\tau)}$ where $f(x) = \sum_{i=0}^d c_i x^i$.

In order for this extractor to be able to extract the actual witness of the relation, we need two things:
\begin{enumerate}
    \item $(f(x))^2 - 1$ is divisible by $t(x)$, which can be reduced to the $d$-Target Group Strong Diffie-Hellman assumption.
    \item $f_{wit}(x) := f(x) - v_0(x) - \sum_{i=1}^\ell a_i v_i(x)$ is in the span of $\{v_i\}_{i=\ell+1}^m$ (equivalently there exist $a_{\ell+1}, \dots, a_m$ such that $f_{wit}(x) = \sum_{i=\ell+1}^m a_i v_i(x)$), which can be reduced to the $d$-Power Diffie-Hellman assumption.
\end{enumerate}
The reductions are based on the fact that the pairing verification equations of the SNARK hold.

Since the two conditions are satisfied the extractor can output $a_{\ell+1}, \dots, a_m$ as a valid witness.

% \let\counterwithout\relax
% \let\counterwithin\relax

% \let\proof\relax
% \let\endproof\relax
% \let\example\relax
% \let\endexample\relax

% %% Tikz configuration
% \usetikzlibrary{shapes.geometric, arrows}

% % Change thanks markers to numbers
% \makeatletter
% \let\@fnsymbol\@arabic
% \makeatother

% %% LLNCS hyperref requirement
% \renewcommand\UrlFont{\color{blue}\rmfamily}

% %% Eurocrypt requirement
% \pagestyle{plain}


% %% Allow align environment to span multiple pages
% \allowdisplaybreaks

% %% Theorem Environments

% % Bold optional theorem titles
% \makeatletter
% \def\th@definition{%
%   \thm@notefont{}% same as heading font
%   \normalfont % body font
% }
% \makeatother

% % No italics
% % \theoremstyle{definition}

% % Rename environments
% \newtheorem{assumption}{Assumption}
% % \newtheorem{construction}{Construction}
% \newtheorem{attention}{Remark}
% \newtheorem{notation}{Notation}

% % Algorithm counter to construction
% \makeatletter 
% \renewcommand\thealgorithm{\theconstruction.\arabic{algorithm}} 
% \@addtoreset{algorithm}{construction}
% \makeatother

% %% Annotations
% \newcommand\todo[1]{\textcolor{red}{TODO: #1}}
% \definecolor{comment-color}{RGB}{68, 59, 141}
% \renewcommand{\algorithmiccomment}[1]{\textcolor{comment-color}{// #1}}
% \newcommand\blue[1]{\textcolor{blue}{#1}}
% \newcommand{\mathhl}[1]{\colorbox{yellow}{$\displaystyle #1$}}



% %% Diagram Macros

% \newcommand{\rightarr}[1]{\sendmessageright{length=2.5cm, top=\colorbox{white}{$#1$}, topstyle={yshift=-9}}}
% \newcommand{\leftarr}[1]{\sendmessageleft{length=2.5cm, top=\colorbox{white}{$#1$}, topstyle={yshift=-9}}}

% \newcommand{\cfbox}[2]{
%     \colorlet{currentcolor}{.}
%     {\color{#1}
%     \fbox{\color{currentcolor}#2}}
% }

\newcommand{\En}{\mathcal{K}}
\newcommand{\Gn}{\mathcal{G}}
\newcommand{\Po}{\mathcal{P}}
\newcommand{\V}{\mathcal{V}}


\newcommand{\niksdef}[6]{
For all expected polynomial-time adversaries 
$\mathcal{P}^*$ 
there exists an expected polynomial-time extractor
$\mathcal{E}$ such that
    \[
    \Pr_{\mathsf{r}}
    \left[
      \begin{array}{l}
        #6
      \end{array}
      \middle\vert
      \begin{array}{l}
        % Generator
        \mathsf{pp} \gets \mathcal{G}(1^{\lambda}, N),\\ 
        % Adversarial Statement, Proof
        (#1, #2, #4) \gets \mathcal{P}^*(\mathsf{pp}, \mathsf{r}),\\
        % Key Generator
        (\pk, \vk) \gets \En(\pp, #1),\\
        % Precondition
        #5,\\
        % Extractor
        #3 \gets \mathcal{E}(\pp, \mathsf{r})
      \end{array}
    \right]
    \approx 
    1
    \]
    where $\mathsf{r}$ denotes an arbitrarily long random tape.
}


% \newcommand{\pp}{\mathsf{pp}}
% \newcommand{\pk}{\mathsf{pk}}
% \newcommand{\vk}{\mathsf{vk}}

\newcommand{\R}{\mathcal{R}}

\newcommand{\FP}{F'}
\newcommand{\io}{\mathsf{x}}
\newcommand{\fu}{\mathsf{u}}
\newcommand{\fw}{\mathsf{w}}
\newcommand{\acc}{\mathsf{U}}
\newcommand{\aw}{\mathsf{W}}
\newcommand{\trivi}{\fu_{\bot}}
\newcommand{\trivw}{\fw_{\bot}}
\newcommand{\fold}{\mathsf{NIFS}}
\newcommand{\snark}{\mathsf{SNARK}}
\newcommand{\RIVC}{\R_{\mathsf{IVC}}}
\newcommand{\Str}{\mathsf{s}}
\newcommand{\com}{\mathsf{com}}

% \newcommand{\Gen}{\mathsf{Gen}}
\newcommand{\Commit}{\mathsf{Commit}}

% \newcommand{\negl}[1]{\mathsf{negl}(#1)}

\section{Recursive Proofs of Knowledge}

In this lecture, 
we discuss recursive zero-knowledge succinct non-interactive
arguments of knowledge (zkSNARKs).
%
We assume familiarity with zkSNARKs.

\subsection{Introduction}

% Zero-Knowledge Proofs
Succinct non-interactive arguments of knowledge are short certificates that attest to the correct execution of a computation without revealing any secret inputs. 
%
Today, 
zero-knowledge proofs are being used 
to secure \emph{billions} of dollars worth of assets~\cite{zerocash, stark}.
%
Zero-knowledge proofs 
enable a new class of secure applications 
with enhanced integrity and privacy guarantees
such as verifiable databases~\cite{zkvsql, vsql, accountablestorage, integridb},
private voting protocols~\cite{privatevoting},
anonymous credentials~\cite{cinderella, dacreds},
and
private cryptocurrencies~\cite{zerocash, pinocchiocoin, stark}.

% Proving a Function
More technically, 
SNARKs 
(for circuit-satisfiability)
allow a prover to demonstrate that it knows a secret witness $w$
such that for prescribed circuit $F$ and prescribed input and output pair $(x, y)$ that $F(x, w) = y$.
%
% Proving Recursion
However, today, we will be interested in
proving \emph{recursive} computation. 
Without loss of generality, we are interested in proving \emph{tail}
recursion,
that is, we want to prove
(unbounded) recursive applications 
of a function $F$.
Unbounded recursion
in general allows us to implement more complex programming patterns such as 
$\mathsf{for}$ and $\mathsf{while}$, 
which are not bounded ahead of time.
This allows us to prove stateful computations with dynamic control flow.
%% Use cases
In practice, 
proving recursion allows us to recursively prove blockchain updates, verifiable delay functions, and even a universal machine,
where each recursive step is a single cycle of a CPU.


% Naive Solution
Historically, 
the best known approach to design a proof system for recursive applications of a function $F$ 
was to unroll the entire execution $F \circ F \circ \cdots \circ F$ into a monolithic arithmetic circuit,
and then use a standard proof system with succinct proofs for circuit satisfiability.
%
Unfortunately,
this would necessarily mean that the prover's space complexity would scale with the \emph{entire trace} of the computation.
%
Moreover, 
in the setting of preprocessed arguments of knowledge
(where the prover and verifier would process the circuit into a succinct key to use for multiple inputs)
this fixed the recursion-depth ahead of time,
which often does not reflect practice.

% Recursive Solution Overview
The first breakthrough was due to Valiant~\cite{valiant} in 2012, 
who proposed incrementally verifiable computation (IVC),
which reflected the recursive structure of the computation
into the proof itself:
%
Given a succinct proof $\pi_{i}$ attesting to $i$ steps of computation,
the prover can write a succinct proof $\pi_{i + 1}$ that attests to $i + 1$ steps
by proving the correct execution of an arithmetic circuit 
that runs the latest step of computation,
and checks $\pi_i$ 
(using the proof system's verifier).
This avoids having to fix the recursion depth ahead of time,
while ensuring that the prover's space complexity only scales with a single step of execution.

Although undoubtedly elegant, 
Valiant's technique introduces a subtle issue:
Proofs of knowledge must satisfy a stronger notion of soundness 
known as \emph{knowledge-soundness}.
%
Essentially,
a proof system is considered knowledge-sound 
if, 
for any successful prover with some secret input to the computation, 
there exists a corresponding \emph{extractor} that,
with at most polynomial overhead,
can retrieve this secret input given access to the ``source code'' of the prover.
%
This extractor-based definition becomes problematic in the recursive setting:
Recursive proofs require \emph{recursive extraction} 
in which the extractor for step $i$ 
plays the successful prover for the extractor at step $i - 1$. 
This incurs a polynomial blowup in the extractor for each successive recursive step. 
In particular, 
this results in a final extractor that runs in exponential-time with respect to the recursion-depth, 
which disqualifies it as a valid extractor. 
%
To account for this issue, 
Valiant's original technique 
(and modern techniques) 
can only provably guarantee logarithmic-depth recursion in standard models.

\subsection{Preliminaries}

We operate in the \textit{preprocessing model}, which means that a trusted party will be responsible for generating a prover and verifier key.

% Definition
\begin{definition}
   [Incrementally Verifiable Computation]\label{def:ivc}
   An 
   incrementally verifiable computation (IVC)  
   scheme is defined by
   PPT algorithms 
   $(\mathcal{G}, \mathcal{P}, \mathcal{V})$ 
   and deterministic $\En$
   denoting the generator, 
   the prover, 
   the verifier,
   and the encoder respectively,
   with the following interface
   \begin{itemize}
     \item $\mathcal{G}(1^\lambda, N) \to \pp$: 
     on input security parameter $\lambda$ and size bounds $N$, 
     samples public parameters $\pp$.
     \item $\En(\pp, F) \to (\pk, \vk)$: 
     on input public parameters $\pp$, 
     and polynomial-time function $F$,
     deterministically produces
     a prover key $\pk$ 
     and a verifier key $\vk$.
     \item $\mathcal{P}(\pk, (i, z_0, z_{i}), \omega_{i}, \pi_{i}) \to \pi_{i+1}$: 
     on input a prover key $\pk$, 
     a counter $i$, 
     an initial input $z_0$, 
     a claimed output after $i$ iterations $z_i$,
     a non-deterministic advice $\omega_i$,
     and an IVC proof $\pi_i$ attesting to $z_i$,
     produces a new proof $\pi_{i + 1}$ attesting to $z_{i + 1} = F(z_{i}, \omega_{i})$.
     \item $\mathcal{V}(\vk, (i, z_0, z_{i}), \pi_{i}) \to \{0, 1\}$: 
     on input a verifier key $\vk$,
     a counter $i$,
     an initial input $z_0$, 
     a claimed output after $i$ iterations $z_i$,
     and an IVC proof $\pi_i$ attesting to $z_i$,
     outputs $1$ if $\pi_i$ is accepting, and 
     $0$ otherwise.
   \end{itemize}
     An IVC scheme 
     $(\mathcal{G}, \En, \mathcal{P}, \mathcal{V})$
     satisfies the following requirements.
     \begin{enumerate}
     \item Perfect Completeness:    
     For any
     PPT adversary $\mathcal{A}$
     \begin{equation*}
     \Pr
     \left[
       \begin{array}{l}
         \mathcal{V}(\vk, (i + 1, z_0, z_{i + 1}), \pi_{i + 1}) = 1 
       \end{array}
       \middle\vert
       \begin{array}{l}
         \mathsf{pp} \gets \mathcal{G}(1^{\lambda}, N),\\
         (F, (i, z_0, z_i), (\omega_i, \pi_i)) \gets \mathcal{A}(\mathsf{pp}),\\
         (\pk, \vk) \gets \En(\pp, F),\\
         z_{i + 1} \gets F(z_{i}, \omega_{i}),\\
         \mathcal{V}(\vk, i, z_0, z_{i}, \pi_{i}) = 1,\\
         \pi_{i + 1} \gets \mathcal{P}(\pk, (i, z_0, z_i), (\omega_{i}, \pi_{i}))
       \end{array}
     \right] = 1
     \end{equation*}
     where $F$ is a polynomial-time computable function represented as an arithmetic circuit.
   \item Knowledge Soundness:
   Consider constant $n \in \mathbb{N}$.

   \niksdef
   % Structure
   {F}
   % Statement
   {(z_0, z_i)}
   % Witness
   {(\omega_0, \ldots, \omega_{n - 1})}
   % Proof Format
   {\Pi}
   % Precondition
   {\mathcal{V}(\vk, (n, z_0, z), \Pi) = 1}
   % Postcondition
   {z_n = z \text{ where }\\ z_{i + 1} \gets F(z_i, \omega_i)\\ \forall i \in \{0, \ldots, n - 1\}}
   
   Moreover, 
     $F$ is a polynomial-time computable function represented as an arithmetic circuit.
     \item Succinctness: 
     The size of an IVC proof $\pi$ is independent of the number of iterations $i$.
   \end{enumerate}
\end{definition}

\begin{definition}[Non-Interactive Argument of Knowledge]\label{def:nark}
   Consider a relation $\R$ over 
   public parameters, structure, instance, and witness tuples.
   A non-interactive argument of knowledge for $\R$ consists of PPT algorithms
   $(\Gn, \Po,\V)$ 
   and deterministic $\En$,
   denoting the generator, 
   the prover,    
   the verifier
   and the encoder respectively with the following interface.
   \begin{itemize}
       \item $\Gn(1^{\lambda}, N) \to \pp$: 
       On input security parameter $\lambda$,
       and length parameter $N$
       samples public parameters $\pp$.
       \item $\En(\pp, \Str) \to (\pk, \vk)$: 
       On input structure $\Str$, 
       representing common structure among instances,
       outputs the prover key $\pk$ and verifier key $\vk$.
       \item $\Po(\pk, u, w) \to \pi$: On input instance $u$ and
         witness $w$, outputs a proof $\pi$ proving that $(\pp, \Str, u, w) \in \R$.
       \item $\V(\vk, u, \pi) \to \{0, 1\}$: 
       Checks instance $u$ 
       given proof $\pi$.
   \end{itemize}
   An argument of knowledge satisfies \textit{completeness} if for any PPT adversary $\mathcal{A}$,
   \begin{align*}
   \Pr
   \left[
       \begin{array}{l}
       \V(\vk, u, \pi) = 1
       \end{array}
       \middle\vert
       \begin{array}{l}
       \pp \gets \Gn(1^{\lambda}, N),\\
       (\Str,(u, w)) \gets \mathcal{A}(\pp),\\
       (\pp, \Str, u, w) \in \R,\\
       (\pk, \vk) \gets \En(\pp, \Str),\\
       \pi \gets \Po(\pk, u, w)
       \end{array}
       \right]
   = 1.
   \end{align*}
   An argument of knowledge satisfies \textit{knowledge soundness} if for all PPT adversaries $\Po^*$ there exists a PPT extractor $\mathcal{E}$ such that for all randomness $\mathsf{r}$
   \[
   \Pr
   \left[
       \begin{array}{l}
       (\pp, \Str, u, w) \in \R
       \end{array}
       \middle\vert
       \begin{array}{l}
       \pp \gets \Gn(1^{\lambda}, N),\\
       (\Str,u,\pi) \gets \Po^*(\pp, \mathsf{r}),\\
       (\pk, \vk) \gets \En(\pp, \Str),\\
       \V(\vk, u, \pi) = 1,\\
       w \gets \mathcal{E}(\pp, \mathsf{r})
       \end{array}
       \right]
   \approx 1.
   \]
\end{definition}

\begin{definition}[Succinctness]
   A non-interactive argument system is succinct if the size of the proof $\pi$
   is polylogarithmic in the size of the witness $w$.
\end{definition}

\begin{definition}[Commitment Scheme]\label{def:commitment}
   A commitment scheme is defined by polynomial-time algorithm
   $\Gen : \mathbb{N}^2 \to P$
   that produces public parameters given the security parameter and size parameter, 
   a deterministic polynomial-time algorithm
   $\Commit : P \times M \times R \to C$
   that produces a commitment in $C$ given a public parameters, message, and randomness tuple
   such that 
   binding holds.
   That is, 
   for any $\mathsf{PPT}$ adversary $\mathcal{A}$,
   given
   $\pp \gets \Gen(\lambda, n)$,
   and given $((m_1, r_1), (m_2, r_2)) \gets \mathcal{A}(\pp)$
   we have that
   \[
       \Pr[(m_1, r_1) \neq (m_2, r_2) \land \Commit(\pp, m_1, r_1) = \Commit(\pp, m_2, r_2)] \approx 0.
   \]
   %
   The commitment scheme is deterministic if $\Commit$ does not use its randomness. 
 \end{definition}

 \begin{definition}[Circuit Satisfiability]
   We define the circuit satisfiability relation $\mathsf{CSAT}$
   over structure, instance, witness tuples
   as follows.
   \begin{equation*}
     \mathsf{CSAT}
     = 
     \left\{
     \begin{array}{l}
         % Statment, Witness
         (C, (x, y), w)
     \end{array}
     \middle\vert
     \begin{array}{l}
         C(x, w) = y
     \end{array}
     \right\}.
   \end{equation*}  
 \end{definition}

\subsection{Construction}

% Formal Construction

\begin{construction}[Incrementally Verifiable Computation]\label{cons:ivc}
   Given a 
   a succinct commitment scheme
   $(\Gen, \Commit)$
   and a
   succinct non-interactive argument of knowledge 
   $\snark$ for circuit-satisfiability
   we construct an IVC scheme as follows.
   
   %% Function F  
   Consider an arithmetic circuit $F$ 
   that takes non-deterministic input
   We begin by defining an augmented circuit
   $\FP$
   as follows,
   where all input arguments are taken as non-deterministic advice.
   %
   \begin{mdframed}[nobreak=true]
     \noindent \underline{$\FP(\io_i, \omega_i, \pi_i)$}:
     \begin{enumerate}
       \item Parse $(\vk_\snark, i, z_0, z_i) \gets \io_i$.
       \item  If $i = 0$:
       \begin{enumerate}
         \item Check that $z_0 = z_i$.
         \label{ivc:fp:base}
       \end{enumerate}
       \item Otherwise:
       \begin{enumerate} 
         \item Check that $\snark.\V(\vk_\snark, \io_i, \pi_i) = 1$.
         \label{ivc:fp:check:general}
       \end{enumerate}
       \item Output $\io_{i + 1} \gets (\vk_{\snark}, i + 1, z_0, F(z_i, \omega_i))$.
       \label{ivc:fp:output}
     \end{enumerate}  
   \end{mdframed} 
   %
   %
   Given the augmented circuit $\FP$,
   we define $(\Gn, \En, \Po, \V)$ as follows.
   \begin{mdframed}[nobreak=true]
     \underline{$\Gn(\lambda, N)$}:
     \begin{enumerate}
       \item Output $\pp \gets \snark.\mathcal{G}({\lambda}, N)$.
     \end{enumerate}
   \end{mdframed}
   %
   \begin{mdframed}[nobreak=true]
     \underline{$\En(\pp, F)$}:
     \begin{enumerate}
       
       \item Compute 
       $(\pk_\snark, \vk_\snark) \gets \snark.\En(\pp, \FP)$.
       \item Output $\pk \gets (\FP, \pk_\snark, \vk_\snark)$ and $\vk \gets \vk_\snark$.
     \end{enumerate}
   \end{mdframed}
   \begin{mdframed}[nobreak=true]
     \underline{$\Po(\pk, (i, z_0, z_i), (\omega_i, \pi_i))$}:
     \begin{enumerate} 
       \item Parse $(\FP, \pk_\snark, \vk_\snark) \gets \pk$.
       \item Compute
       $\io_{i + 1} \gets \FP(\vk_{\snark},
       (i, z_0, z_i), \omega_i, \pi_i)$.
       \item Let $\io_i \gets (\vk_{\snark}, i, z_0, z_i)$
       \label{ivc:prover:io}
       \item Output
       $
       \pi_{i + 1} \gets \snark.\Po(\pk_\snark, (\bot, \io_{i + 1}), (\io_i
       , \omega_i, \pi_i))
       $.
       \label{ivc:prover:proof}
     \end{enumerate} 
   \end{mdframed}
   %
   \begin{mdframed}[nobreak=true]
     \underline{$\V(\vk, (i, z_0, z_i), \pi_i)$}:
     \begin{enumerate}
       \item If $i = 0$: Check that $z_i = z_0$.
       \label{ivc:verifier:base}
       \item Otherwise:
       \begin{enumerate}
         \item Parse $\vk_{\snark} \gets \vk$.
         \item Let $\io_i \gets (\vk_\snark, i, z_0, z_i)$.
         \label{ivc:v:check:first}
         \item Check that
         $\snark.\V(\vk_\snark, (\bot, \io_i), \pi_i) = 1$.
         \label{ivc:v:check:second}
       \end{enumerate}
       
     \end{enumerate}
   \end{mdframed}
 \end{construction}

\begin{lemma}[Completeness]
 Construction~\ref{cons:ivc}
 is complete.
\end{lemma}
% \newcommand{\proof}{\noindent{\bf Proof. }} %% To begin a proof write \proof

\begin{proof}
   % Adversary
 Consider arbitrary PPT adversary $\mathcal{A}$.
 % public parameters
 Suppose $\pp \gets \Gen(1^{\lambda}, N)$.
 % Structure, Instance, Witness
 Suppose that
 \[
 (F, (z_0, z_i, i), \pi_i) \gets \mathcal{A}(\pp).
 \]
 % Prover Key, Verifier Key
 Suppose that for
   $(\pk, \vk) \gets \En(\pp, F)$
 % Precondition
 we have that
 \begin{align}\label{ivc:completeness:precondition:verifier}
   \V(\vk, (z_0, z_{i}, i), \pi_{i}) = 1.
 \end{align}
 % Postcondition
 Then,
 given
 \begin{align*}
   z_{i + 1} \gets F(z_{i}, \omega_{i})
 \end{align*}
 and 
 \begin{align*}
   \pi_{i + 1} \gets \Po(\pk, (z_0, z_i, i), (\omega_i, \pi_i))
 \end{align*}
 we must show that
 \begin{align}\label{ivc:completeness:postcondition}
   \V(\vk, (z_0, z_{i+1}, N), \pi_{i + 1}) = 1
 \end{align}
 with probability $1$.
 %
 
 % Case i = 0
 Indeed, 
 consider the base case where $i = 0$.
 %
 Then, 
 by Precondition~\ref{ivc:completeness:precondition:verifier},
 by the verifier's check in the base case 
 (Step~\ref{ivc:verifier:base})
 we have that $z_0 = z_i$.
 %
 Therefore,
 $\Po$ can successfully compute $\io_{i + 1}$ 
 (Step~\ref{ivc:prover:io}),
 because the base case check of $\FP$ 
 (Step~\ref{ivc:fp:base}) passes.
 %
 Then,
 by construction of $\FP$
 (Step~\ref{ivc:fp:output})
 we have that 
 \[
   \io_{i + 1} = (\vk_{\snark}, i + 1, z_0, F(z_i, \omega_i))
 \]
 Moreover,
 by the completeness
 of $\snark$,
 we have that $\pi_{i + 1}$ generated by $\Po$ 
 (Step~\ref{ivc:prover:proof})
 is indeed satisfying.
 %
 Therefore, 
 both the checks of $\V$ in Steps~\ref{ivc:v:check:first} and~\ref{ivc:v:check:second}
 are passing.
 %
 As such,
 we have that postcondition~\ref{ivc:completeness:postcondition}
 holds.
 
 
 % Case i \geq 1
 Suppose instead that $i \geq 1$.
 %
 by Precondition~\ref{ivc:completeness:precondition:verifier},
 by the verifier's check in the general case 
 we have that
 \begin{align*}
   \snark.\V(\vk_\snark, (\bot, \io_i), \pi_i) = 1
 \end{align*}
 for $\io_i = (\vk_\snark, i, z_0, z_i)$.
 %
 Then,
 $\Po$ can successfully compute $\io_{i + 1}$ 
 (Step~\ref{ivc:prover:io}),
 as the SNARK verifier check in $\FP$
 (Step~\ref{ivc:fp:check:general})
 holds.
 %
 %
 Once again,
 by construction of $\FP$
 (Step~\ref{ivc:fp:output})
 we have that 
 \[
   \io_{i + 1} = (\vk_{\snark}, i + 1, z_0, F(z_i, \omega_i))
 \]
 Moreover,
 by the completeness
 of $\snark$,
 we have that $\pi_{i + 1}$ generated by $\Po$ 
 (Step~\ref{ivc:prover:proof})
 is indeed satisfying.
 %
 Therefore, 
 both the checks of $\V$ in Steps~\ref{ivc:v:check:first} and~\ref{ivc:v:check:second}
 are passing.
 %
 As such,
 we have that postcondition~\ref{ivc:completeness:postcondition}
 holds.
\end{proof}

\begin{lemma}[Knowledge-Soundness]
 Construction~\ref{cons:ivc}
 is knowledge-sound.
\end{lemma}

% Proof for ivc knowledge soundness

\begin{proof}
   % Premise
   Let $n$ be a global constant.
   %
   Consider a deterministic expected polynomial-time adversary $\Po^*$.
   %
   Let $\pp \gets \Gen(1^{\lambda}, N)$.
   %
   Suppose 
   on input $\pp$ and randomness $\mathsf{r}$,
   $\Po^*$
   outputs 
   polynomial-time function $F$,
   instance $(z_0, z)$,
   and IVC proof $\pi$. 
   %
   Let $(\pk, \vk) \gets \En(\pp, F)$.
   Suppose that
   % Precondition
   \begin{align*}
     \V(\vk, (z_0, z, n), \pi) = 1.
   \end{align*}
   % Postcondition
   We must construct an expected polynomial-time extractor $\mathcal{E}$
   that, 
   with input $(\pp, \mathsf{r})$,
   outputs $(\omega_0, \ldots, \omega_{n - 1})$ 
   such that by computing
   \begin{align*}
     z_{i + 1} \gets F(z_{i}, \omega_{i})
   \end{align*}
   we have that $z_n = z$ with probability 
   $1 - \negl{\lambda}$.
   
   % Overview
   We show inductively that we can construct an expected polynomial-time extractor $\mathcal{E}_i$ that outputs 
   $((z_i, \ldots, z_{n - 1}), (\omega_i, \ldots, \omega_{n - 1}), \pi_i)$ such that for all $j \in  \{i + 1, \ldots, n\}$,
   \begin{align*}
     z_j = F(z_{j - 1}, \omega_{j - 1})
   \end{align*}
   and 
   \begin{align}\label{eq:ih:2}
     \V(\vk, z_0, z_i, \pi_i) = 1
   \end{align}
   for $z_n = z$ with probability $1 - \negl{\lambda}$.
   %
   Then, 
   because when $i = 0$,
   $\V$ checks that $z_0 = z_i$ 
   the values $(\omega_0, \ldots, \omega_{n - 1})$ 
   retrieved by $\mathcal{E} = \mathcal{E}_0$
   are such that computing  
   $z_{i + 1} = F(z_i, \omega_i)$ for all $i \geq 1$ gives $z_n = z$.
   
   At a high level,
   to construct an extractor $\mathcal{E}_{i - 1}$,
   we first assume the existence of $\mathcal{E}_i$ that satisfies the inductive hypothesis. 
   We then use $\mathcal{E}_i$ to construct an adversary 
   for the underlying succinct non-interactive argument, which we denote as $\widetilde{\Po}_{i - 1}$.
   This in turn guarantees an extractor for the underlying non-interactive argument,
   which we denote as $\widetilde{\mathcal{E}}_{i - 1}$. 
   We then use $\widetilde{\mathcal{E}}_{i - 1}$ to construct $\mathcal{E}_{i - 1}$ that satisfies the inductive hypothesis.
   
   
   % Base case
   In the base case,
   let $\mathcal{E}_n(\pp, \mathsf{r})$ 
   output 
   $(\bot, \bot, \pi_n)$ 
   where $\pi_n$ is the output of 
   $\Po^*(\pp, \mathsf{r})$.
   By the premise,
   we have that $\pi_n$ is satisfying.
   As such,
   $\mathcal{E}_n$ succeeds with probability $1 - \negl{\lambda}$ in expected polynomial-time.
   
   % Inductive Step
   For $i \geq 1$, 
   suppose we can construct an expected polynomial-time extractor 
   $\mathcal{E}_i$ 
   that outputs
   $((z_i, \ldots, z_{n - 1}), (\omega_i, \ldots, \omega_{n - 1}))$, 
   and $\pi_i$ 
   that satisfies the inductive hypothesis.
   % Construct SNARK prover
   To construct an extractor $\mathcal{E}_{i - 1}$, 
   we first construct an adversary $\widetilde{\Po}_{i - 1}$
   for the underlying SNARK as follows.
   %
   %
   %
   \begin{mdframed}[nobreak=true]
     \noindent \underline{$\widetilde{\Po}_{i - 1}(\pp, \mathsf{r})$}: 
     \begin{enumerate}
       \item Let $(F, z_0) \gets \Po^*(\pp, \mathsf{r})$
       \item Let $((z_i, \ldots, z_{n - 1}), (\omega_i, \ldots, \omega_{n -
       1}), \pi_i) \gets \mathcal{E}_i(\pp, \mathsf{r})$.
       \item Let $\vk_{\snark} \gets \snark.\En(\pp, F)$.
       \item Let $\io_i \gets (\vk_\snark, (i, z_0, z_i))$.
       \item Output $(\FP, (\bot, \io_i), \pi_i)$.
     \end{enumerate}
   \end{mdframed}
   
   % SNARK prover success probability
   We now analyze the success probability of $\widetilde{\Po}_{i - 1}$. 
   By the inductive hypothesis,
   we have that 
   \[
   \V(\vk, z_0, z_i, \pi_i) = 1,
   \]
   where 
   $\pi_i \gets \mathcal{E}_i(\pp, \mathsf{r})$ 
   with probability 
   $1 - \negl{\lambda}$.
   Therefore, 
   by the the verifier's checks 
   we have that
   \[
   \snark.\V(\vk_\snark, (\bot, \io_i), \pi_i) = 1
   \]
   for $\io_i = (\vk_\snark, (i, z_0, z_i))$
   and $\vk_\snark \gets \snark.\En(\pp, \FP)$.
   Therefore,
   $\widetilde{\Po}_{i - 1}$
   succeeds in producing a satisfying proof $\pi_i$
   for structure $\FP$ and instance $\io_i$
   with probability $1 - \negl{\lambda}$.
 
   
   % Corresponding SNARK extractor
   Then,
   by the knowledge soundness of $\snark$
   there exists an 
   expected-polynomial-time
   extractor $\widetilde{\mathcal{E}}_{i - 1}$ that
   outputs 
   $(\io_{i - 1}, \omega_{i - 1}, \pi_{i - 1})$
   such that
   $\io_i = \FP(\io_{i - 1}, \omega_{i - 1}, \pi_{i - 1})$
   with probability $1 - \negl{\lambda}$.
   
   % Constructing E_{i - 1}
   Given $\widetilde{\mathcal{E}}_{i - 1}$, 
   we construct an expected polynomial time $\mathcal{E}_{i - 1}$ as follows.
 
 
   \begin{mdframed}[nobreak=true]
   \noindent \underline{$\mathcal{E}_{i - 1}(\pp, \mathsf{r})$}: 
   \begin{enumerate}
     \item Run $\widetilde{\Po}_{i - 1}(\pp, \mathsf{r})$
     to parse
     \[
     ((z_i, \ldots, z_{n - 1}), (\omega_i, \ldots, \omega_{n -
     1}), \pi_i)
     \]
     from its internal state.
     \item Compute $(\io_{i - 1}, \omega_{i - 1}, \pi_{i - 1}) \gets \widetilde{\mathcal{E}}_{i - 1}(\pp, \mathsf{r})$.
     \item Parse $(\vk_\snark, (i - 1, z_0, z_{i - 1})) \gets \io_{i - 1}$
     \item Output $((z_{i - 1}, \ldots, z_{n - 1}), (\omega_{i - 1}, \ldots, \omega_{n - 1}), \pi_{i - 1})$. 
   \end{enumerate}
 \end{mdframed}
 
  We now reason about the success probability of $\mathcal{E}_{i - 1}$.
   We first reason that the output $(z_{i - 1}, \ldots, z_{n - 1})$, and
   $(\omega_{i - 1}, \ldots, \omega_{n - 1})$ are valid.
   By the inductive hypothesis, 
   we already have that for all $j \in \{i + 1, \ldots, n\}$,
   \begin{align*}
     z_j = F(z_{j - 1}, \omega_{j - 1})
   \end{align*}
   with probability $1 - \negl{\lambda}$.
   Moreover,
   by the success probability of $\mathcal{E}_{i - 1}$,
   we have that
   \begin{align*}
     z_{i} = F(z_{i - 1}, \omega_{i - 1})
   \end{align*}
   and that
   \[
   \snark.\V(\vk_\snark, \io_{i - 1}, \pi_{i - 1}) = 1
   \]
   where $\io_{i - 1} = (\vk_\snark, (i - 1, z_0, z_i))$
   with probability $1 - \negl{\lambda}$.
   %
   Therefore we have that $\mathcal{E}_{i - 1}$
   succeeds with probability $1 - \negl{\lambda}$
   satisfying the inductive hypothesis.
 \end{proof} 
\newcommand{\proofsketch}{\smallskip\noindent{\bf Proof sketch. }}
\algrenewcommand\algorithmicfunction{\textbf{Machine}}
\newcommand{\bits}{\set{0,1}}
\newcommand{\Ex}{\mathbb{E}}

\renewcommand{\O}{\ensuremath{\mathcal{O}}}
\newcommand{\To}{\rightarrow}
\newcommand{\e}{\epsilon}
% \newcommand{\R}{\mathbb{R}}
\newcommand{\N}{\mathbb{N}}
\newcommand{\Z}{\mathbb{Z}}
\newcommand{\logAnd}{\wedge}

\newcommand{\indis}{\mathrel{\overset{\makebox[0pt]{\mbox{\normalfont\tiny c}}}{\approx}}}
\newcommand{\allindis}{\mathrel{\overset{\makebox[0pt]{\mbox{\normalfont\tiny p/s/c}}}{\approx}}}

\newcommand{\cclass}[1]{\mathsf{#1}}
\renewcommand{\P}{\cclass{P}}
\newcommand{\NP}{\cclass{NP}}
\newcommand{\Time}{\cclass{Time}}
\newcommand{\BPP}{\cclass{BPP}}
\newcommand{\Size}{\cclass{Size}}
\newcommand{\Ppoly}{\cclass{P_{/poly}}}
\newcommand{\CSAT}{\ensuremath{\mathsf{CSAT}}}
\newcommand{\SAT}{\ensuremath{\mathsf{3SAT}}}
\newcommand{\IS}{\mathsf{INDSET}}



\newcommand{\inp}{\mathsf{in}}
\newcommand{\outp}{\mathsf{out}}

\newcommand{\Param}{\kappa}
\newcommand{\Adv}{\mathsf{Adv}}
\newcommand{\Supp}{\mathsf{Supp}}


\newcommand{\PRG}{\mathsf{G}}
\renewcommand{\Enc}{\mathsf{Enc}}
\renewcommand{\Dec}{\mathsf{Dec}}
\renewcommand{\sk}{\mathsf{sk}}
\newcommand{\sfC}{\mathsf{C}}
\newcommand{\sfR}{\mathsf{R}}

\newcommand{\eqdef}{\stackrel{\text{\tiny def}}{=}}

\newcommand{\cF}{\mathcal{F}}

\newcommand{\angles}[1]{\langle #1 \rangle}
\newcommand{\iprod}[1]{\angles{#1}}

\newcommand{\Com}{\mathsf{Com}}


% Real vs. Ideal
\newcommand{\RealAdv}{\mathcal{A}}
\newcommand{\IdealAdv}{\mathcal{S}}
\newcommand{\RealVar}{\mathsf{Real}}
\newcommand{\IdealVar}{\mathsf{Ideal}}

% Participating parties
\newcommand{\PartyA}{P_1}
\newcommand{\PartyB}{P_2}
\newcommand{\InputA}{x_1}
\newcommand{\InputB}{x_2}


% Garbling Schemes
\newcommand{\Garble}{\mathsf{Garble}}
\newcommand{\Cir}{C}
\newcommand{\GCir}{\widetilde{C}}
\newcommand{\Lab}{\mathsf{lab}}

% Proof
\newcommand{\Sim}{\mathsf{Sim}}

% GMW

% Misc
\newcommand{\out}{\mathsf{out}}
\newcommand{\Assign}{:=}

\chapter{Secure Computation}

\section{Introduction}
Secure multiparty computation considers the problem of different parties
computing a joint function of their separate, private inputs without revealing
any extra information about these inputs than that is leaked by just the result
of the computation. This setting is well motivated, and captures many different
applications. Considering some of these applications will provide intuition
about how security should be defined for secure computation:
\begin{description}
  \item[Voting:] Electronic voting can be thought of as a multi party computation
	  between $n$ players: the voters. Their input is their choice $b \in \{0,1\}$
    (we restrict ourselves to the binary choice setting without loss of generality), and the function
    they wish to compute is the majority function.

    Now consider what happens when only one user votes: their input is trivially
    revealed as the output of the computation. What does privacy of inputs mean
    in this scenario?

  \item[Searchable Encryption:] Searchable encryption schemes allow clients
    to store their data with a server, and subsequently grant servers tokens
    to conduct specific searches. However, most schemes do not consider access
    pattern leakage. This leakage tells the server potentially valuable information
    about the underlying plaintext. How do we model all the different kinds
    information that is leaked?
\end{description}

From these examples we see that defining security is tricky, with lots of
potential edge cases to consider. We want to ensure that no party can learn
anything more from the secure computation protocol than it can from just its
input and the result of the computation. To formalize this, we adopt the
\textbf{real/ideal paradigm}.



\section{Real/Ideal Paradigm}
\paragraph{Notation.} 
Suppose there are $n$ parties, and party $P_i$ has access to some data $x_i$. They are trying to compute some function of their inputs $f(x_1, \dotsc, x_n)$. The goal is to do this securely: even if some parties are corrupted, no one should learn more than is strictly necessitated by the computation.

\paragraph{Real World.} In the real world, the $n$ parties execute a protocol $\Pi$
to compute the function $f$. This protocol can involve multiple rounds of
interaction. %Each party can additionally have some randomness.
The real world adversary $\RealAdv$ can corrupt arbitrarily many (but not all) parties.


\paragraph{Ideal World.} In the ideal world, an angel helps in the
computation of $f$:
each party sends their input to the angel and receives the output of the computation $f(x_1, \dotsc, x_n)$.
Here the ideal world adversary $\IdealAdv$ can again corrupt arbitrarily many (but not all) parties.

To model malicious adversaries, we need to modify the ideal world model as follows. 
Some parties are honest, and each honest party $P_i$ simply sends $x_i$ to the angel. The other parties are corrupted and are under control of the adversary $\IdealAdv$. The adversary chooses an input $x_i'$ for each corrupted party $P_i$ (where possibly $x_i' \neq x_i$) and that party then sends $x_i'$ to the angel. The angel computes a function $f$ of the values she receives (for example, if only party 1 is honest, then the angel computes $f(x_1, x_2', x_3', \dotsc, x_n')$) in order to obtain a tuple $(y_1, \dotsc, y_n)$. 
She then sends $y_i$ of corrupted parties to the adversary, who gets to decide whether or not honest parties will receive their response from the angel. The angel obliges. Each honest party $P_i$ then outputs $y_i$ if they receive $y_i$ from the angel and $\perp$ otherwise, and corrupted parties output whatever the adversary tells them to. 

 



\paragraph{Definition of Security.} 
A protocol $\Pi$ is secure against computationally bounded adversaries if for every PPT adversary $\RealAdv$ in the real world, there exists an PPT adversary $\IdealAdv$ in the ideal world such that for all tuples of bit strings $(x_1, \dotsc, x_n)$, we have
\[ \mathrm{Real}_{\Pi, \RealAdv}(x_1, \dotsc x_n) \stackrel{c}{\simeq} \mathrm{Ideal}_{F,\IdealAdv}(x_1, \dotsc, x_n) \]
where the left-hand side denotes the output distribution induced by $\Pi$ running with $\RealAdv$, and the right-hand side denotes the output distribution induced by running the ideal protocol $F$ with $\IdealAdv$. 
The ideal protocol is either the original one described for semi-honest adversaries, or the modified one described for malicious adversaries. 





%We require that the views of the parties
%in each of the scenarios be identical, i.e.\ that a real-world execution of the
%protocol $\Pi$ should not leak any information not leaked by the ideal-world
%execution. Hence, the parties can only learn what they can infer from their
%inputs and the output $f(\InputA, \InputB)$. More formally, assuming $\RealAdv$
%corrupts one party (say $\PartyA$, wlog), we define random variables
%$\RealVar_{\Pi, \RealAdv}(\InputA, \InputB) = \RealAdv(\InputA, r_1, \text{messages
%sent in } \Pi)$ and $\IdealVar_{F, \IdealAdv}(\InputA, \InputB) = \IdealAdv(\InputA,
%f(\InputA,\InputB))$.  These random variables represent the views of the
%adversary in each of the two settings. Our definition of security thus requires
%that
%
%\begin{equation*}
%\RealVar_{\Pi, \RealAdv}(\InputA, \InputB) \indis \IdealVar_{F, \IdealAdv}(x_1, x_2).
%\end{equation*}

\paragraph{Assumptions.} We have brushed over some details of the above setting.
Below we state these assumptions explicitly:
\begin{enumerate}
  \item \textbf{Communication channel:} We assume that the communication channel
    between the involved parties is completely insecure, i.e., it does not preserve
    the privacy of the messages. However, we assume that it is reliable, which means
    that the adversary can drop messages, but if a message is delivered, then
    the receiver knows the origin.

  \item \textbf{Corruption model:} We have different models of how and when the
    adversary can corrupt parties involved in the protocol:
    \begin{itemize}
      \item
        \emph{Static:} The adversary chooses which parties to corrupt before the
        protocol execution starts, and during the protocol, the malicious parties
        remain fixed.
      \item
        \emph{Adaptive:} The adversary can corrupt parties dynamically during
        the protocol execution, but the simulator can do the same.
      \item
        \emph{Mobile:} Parties corrupted by the adversary can be ``uncorrupted''
        at any time during the protocol execution at the adversary's discretion.
    \end{itemize}

  \item \textbf{Fairness:} The protocols we consider are not ``fair'', i.e.,
    the adversary can cause corrupted parties to abort arbitrarily. This can
    mean that one party does not get its share of the output of the computation.

  \item \textbf{Bounds on corruption:} In some scenarios, we place upper bounds
    on the number of parties that the adversary can corrupt.

  \item \textbf{Power of the adversary:} We consider primarily two types of
    adversaries:
    \begin{itemize}
      \item \emph{Semi-honest adversaries:} Corrupted parties follow the protocol
        execution $\Pi$ honestly, but attempt to learn as much information as they
        can from the protocol transcript.

      \item \emph{Malicious adversaries:} Corrupted parties can deviate arbitrarily
        from the protocol $\Pi$.
    \end{itemize}

  \item \textbf{Standalone vs.\ Multiple execution:} In some settings, protocols
    can be executed in isolation; only one instance of a particular protocol
    is ever executed at any given time. In other settings, many different protocols
    can be executed concurrently. This can compromise security.
\end{enumerate}







\section{Oblivious transfer}

\emph{Rabin's oblivious transfer} sets out to accomplish the following special task of two-party secure computation. The sender has a bit $s \in \{0,1\}$. She places the bit in a box. Then the box reveals the bit to the receiver with probability 1/2, and reveals $\perp$ to the receiver with probability 1/2. The sender cannot know whether the receiver received $s$ or $\perp$, and the receiver cannot have any information about $s$ if they receive $\perp$.

\subsection{1-out-of-2 oblivious transfer}
\emph{1-out-of-2 oblivious transfer} sets out to accomplish the following related task. The sender has two bits $s_0, s_1 \in \{0,1\}$ and the receiver has a bit $c \in \{0,1\}$. The sender places the pair $(s_0, s_1)$ into a box, and the receiver places $c$ into the same box. The box then reveals $s_c$ to the receiver, and reveals $\perp$ to the sender (in order to inform the sender that the receiver has placed his bit $c$ into the box and has been shown $s_c$). The sender cannot know which of her bits the receiver received, and the receiver cannot know anything about $s_{1-c}$.

\begin{lemma}
A system implementing 1-out-of-2 oblivious transfer can be used to implement Rabin's oblivious transfer.
\end{lemma}

\proof
The sender has a bit $s$. She randomly samples a bit $b \in \{0,1\}$ and $r \in \{0,1\}$, and the receiver randomly samples a bit $c \in \{0,1\}$. If $b = 0$, the sender defines $s_0 = s$ and $s_1 = r$, and otherwise, if $b = 1$, she defines $s_0 = r$ and $s_1 = s$. She then places the pair $(s_0, s_1)$ into the 1-out-of-2 oblivious transfer box. The receiver places his bit $c$ into the same box, and then the box reveals $s_c$ to him and $\perp$ to the sender. Notice that if $b = c$, then $s_c = s$, and otherwise $s_c = r$. Once $\perp$ is revealed to the sender, she sends $b$ to the receiver. The recieiver checks whether or not $b = c$. If $b = c$, then he knows that the bit revealed to him was $s$. Otherwise, he knows that the bit revealed to him was the nonsense bit $r$ and he regards it as $\perp$. \\

It is easy to see that this procedure satisfies the security requirements of Rabin's oblivious transfer protocol. Indeed, as we saw above, $s_c = s$ if and only if $b = c$, and since the sender knows $b$, we see that knowledge of whether or not the bit $s_c$ received by the receiver is equal to $s$ is equivalent to knowledge of $c$, and the security requirements of 1-out-of-2 oblivious transfer prevent the sender from knowing $c$. Also, if the receiver receives $r$ (or, equivalently, $\perp$), then knowledge of $s$ is knowledge of the bit that was not revealed to him by the box, which is again prevented by the security requirements of 1-out-of-2 oblivious transfer.  $\qed$

\begin{lemma}
A system implementing Rabin's oblivious transfer can be used to implement 1-out-of-2 oblivious transfer.
\end{lemma}

\proofsketch
The sender has two bits $s_0, s_1 \in \{0,1\}$ and the receiver has a single bit $c$. The sender randomly samples $3n$ random bits $x_1, \dotsc, x_{3n} \in \{0,1\}$. Each bit is placed into its own a Rabin oblivious transfer box. The $i$th box then reveals either $x_i$ or else $\perp$ to the receiver. Let 
\[ S := \{i \in \{1, \dotsc, 3n\} : \text{the receiver knows } x_i\}. \]
The receiver picks two sets $I_0, I_1 \subseteq \{1, \dotsc, 3n\}$ such that $\# I_0 = \# I_1 = n$, $I_c \subseteq S$ and $I_{1-c} \subseteq \{1, \dotsc, 3n\} \setminus S$. This is possible except with probability negligible in $n$. He then sends the pair $(I_0, I_1)$ to the sender. The sender then computes $t_j= \left(\bigoplus_{i \in I_j}x_i \right) \oplus s_j$ for both $j \in \{0,1\}$ and sends $(t_0, t_1)$ to the receiver. \\

Notice that the receiver can uncover $s_c$ from $t_c$ since he knows $x_i$ for all $i \in I_c$, but cannot uncover $s_{1-c}$. One can show that the security requirement of Rabin's oblivious transfer implies that this system satisfies the security requirement necessary for 1-out-of-2 oblivious transfer. $\qed$ \\

We will see below that length-preserving one-way trapdoor permutations can be used to realize 1-out-of-2 oblivious transfer. 

\begin{theorem}
The following protocol realizes 1-out-of-2 oblivious transfer in the presence of computationally bounded and semi-honest adversaries. 
\begin{enumerate}
\item The sender, who has two bits $s_0$ and $s_1$, samples a random length-preserving one-way trapdoor permutation $(f, f^{-1})$ and sends $f$ to the receiver.  Let $b(\cdot)$ be a hard-core bit for $f$.
\item The receiver, who has a bit $c$, randomly samples an $n$-bit string $x_c \in \{0,1\}^n$ and computes $y_c = f(x_c)$. He then samples another random $n$-bit string $y_{1-c} \in \{0,1\}^n$, and then sends $(y_0, y_1)$ to the sender.
\item The sender computes $x_0 := f^{-1}(y_0)$ and $x_1 := f^{-1}(y_1)$. She computes $b_0 := b(x_0) \oplus s_0$ and $b_1 := b(x_1) \oplus s_1$, and then sends the pair $(b_0, b_1)$ to the receiver.
\item The receiver knows $c$ and $x_c$, and can therefore compute $s_c = b_c \oplus b(x_c)$. 
\end{enumerate}
\end{theorem}
\proof
Correctness is clear from the protocol.	
For security, from the sender side, since $f$ is a length-preserving permutation, $(y_0, y_1)$ is statistically indistinguishable from two random strings, hence she can't learn anything about $c$.
From the receiver side, guessing $s_{1-c}$ correctly is equivalent to guessing the hard-core bit for $y_{1-c}$.
\qed


\subsection{1-out-of-4 oblivious transfer}
  We describe how to implement a 1-out-of-4 OT using 1-out-of-2 OT:\@
  \begin{enumerate}
    \item
      The sender, $\PartyA$ samples 5 random values $S_i \gets \bits, i \in \set{1,\dotsc, 5}$.
    \item
      $\PartyA$ computes
      \begin{align*}
        \alpha_0 &= S_0 \xor S_2 \xor m_0\\
        \alpha_1 &= S_0 \xor S_3 \xor m_1\\
        \alpha_2 &= S_1 \xor S_4 \xor m_2\\
        \alpha_3 &= S_1 \xor S_5 \xor m_3
      \end{align*}
      It sends these values to $\PartyB$.
    \item
      The parties engage in 3 1-out-of-2 Oblivious Transfer protocols for the following
      messages: $(S_0, S_1)$, $(S_2, S_3)$, $(S_4, S_5)$. THe receiver's input for
      the first OT is the first choice bit, and for the second and third ones is
      the second choice bit.
    \item
      The receiver can only decrypt one ciphertext.
  \end{enumerate}


\section{Yao's Garbled Circuit}


%\input{HWsolution.tex}
%\part{Yao's Garbled Circuit}

% ===========
\section{Setup}


Yao's Garbled Circuits is presented as a solution to Yao's Millionaires' problem, 
which asks whether 
two millionaires can compete for bragging rights of which is richer
without revealing their wealth to each other. 
It started the area of secure computation. 
We will present a solution for the two party problem;
it can be extended to a polynomial number of parties,
using the techniques from last lecture.

The solution we saw previously needed an interaction for each AND gate.
Yao's solution requires only one message,
so it provides a constant size of interaction.
We present a solution only for semi honest security. 
This can be amplified to malicious security, 
but there are more efficient ways of amplifying this than what we saw last lecture.

\subsection{Secure Computation}

Recall our definition of secure computation. 
We define ideal and real worlds. 
Security is defined to hold if 
anything an attacker can achieve in the real world 
 can also be achieved by an ideal attacker in the ideal world. 
We define the ideal world to have the properties that we desire. 
For security to hold these properties must also hold in the real world.

\subsection{$(\Garble, \Eval)$}
We will provide a definition, similar to how we define encryption, that allows us avoid dealing with simulators all the time. 


Yao's Garbled Circuit is defined as two efficient algorithms $(\Garble, \Eval)$. Let the circuit $C$ have $n$ input wires.
$\Garble$ produces the garbled circuit and two labels for each input wire. The labels are for each of 0 and 1 on that wire and are like encryption keys. 

\[
(\tilde{C}, \{\ell_{i,b}\}_{i \in [n], b \in \{0,1\}}) \leftarrow \Garble(1^k, C) 
\]

To evaluate the circuit on a single input we must choose a value for each of the n input wires.
Given n of 2n input keys, $\Eval$ can evaluate the circuit on those keys and get the circuit result.
\[
C(x) \leftarrow \Eval(\tilde{C}, \{\ell_{i, x_i}\}_{i \in [n]}) 
\]

\paragraph{Correctness}
Correctness is as usual, if you garble honestly, evaluation should produce the correct result. 
\[
\forall C, x 
Pr[ C(x) = \Eval(\tilde{C}, \{l_{i, x_i}\}),  (\tilde{C}, \{\ell_{i,b}\}) = \Garble(1^k, C)] = 1 - neg(k)
\]


\paragraph{Security}
For security we require that a party receiving 
a garbled circuit and n inputs labels 
can not computationally distinguish the joint distribution of the circuits and labels
from the distribution produced by 
a simulator with access to the circuit and its evaluation on the input that the labels represent. 
The simulator does not have access to the actual inputs.
If this holds, the party receiving the garbled circuit and n labels can not determine the inputs.

\begin{align*}
&\exists \Sim : \forall C, x\\
&(\tilde{C}, \{\ell_{i,x_i}\}_{i \in [n]}) \simeq \Sim(1^k, C, C(x)) \text{ where} \\
&(\tilde{C}, \{\ell_{i,b}\}_{i \in [n], b \in \{0,1\}}) \leftarrow \Garble(1^k, C) 
\end{align*}

For simplicity we pass the circuit to the simulator.
You could also use universal circuits and pass 
in with the inputs the specific circuit that the universal circuit should realize. 



\section{Use for Semi-honest two party secure communication}
Alice, with input $x^1$, and Bob, with input $x^2$, have a circuit, C, that they want to evaluate securely. 
The size of their combined inputs is n, so $|x^1| = n_1, |x^2| = n - n_1, |x^1| + |x^2| = n$.
They can do this by Alice garbling a circuit and sending input wire labels to Bob, as in Figure \ref{fig:message}.

Alice garbles the circuit and passes it to Bob, $\tilde{C}$.
Alice passes the labels for her input directly to Bob, $\{\ell_{i, x^1_i}\}_{i \in [n] / [n_2]}$.
Alice passes all the labels for Bob's input wires into oblivious transfer, $\{\ell_{i, b_i}\}_{i \in [n] / [n_1], b \in \{0,1\}}$, 
from which Bob can retrieve the labels for his actual inputs, $\{\ell_{i, x^2_i}\}_{i \in [n] / [n_1]}$.
Bob now has the garbled circuit and one label for each input wire. 
He evaluates the garbled circuit on those garbled inputs and learns $C(x^1||x^2)$.
Bob does not learn anything besides the result as he has only the garbled circuit and n garbled inputs.
Alice does not learn anything as she uses oblivious transfer to give Bob his input labels and receives nothing in reply.

\begin{figure}[htbp]
\begin{center}
\setlength{\unitlength}{1cm}
\begin{picture}(10, 7)(-5, -4)
% \put(-.5,2){\makebox(1,1){C}}
 \put(-6,2){\makebox{Alice: $C, x^1$}}
 \put(-6,1.3){\makebox{$(\tilde{C}, \{\ell_{i,b}\}) \leftarrow \Garble$}}
 \put(4,2){\makebox{Bob: $C, x^2$}}

 \put(-1,0){\makebox(2,2){$\underrightarrow{\tilde{C}}$}}
  \put(-1,-0.8){\makebox(2,2){$\underrightarrow{ S_{out}^0 \text{ is 0 }, S_{out}^1 \text{ is 1 } }$}}


 \put(-1,-2){\makebox(2,2){$\underrightarrow{\{\ell_{i, x^1_i}\}_{i \in [n] / [n_2]}}$}}
% \put(-1,-1){\makebox(2,2){$\underrightarrow{\ell_{i,0}, \, \ell_{i,1} \forall i \in  [n]/[n_1] }$}}

 \put(-.5,-3){\framebox(1,1){OT}}
  \put(-1,-2.8){\line(1,0){.5}}
   \put(-1.6,-2.8){\makebox{$\ell_{i,1}$}}

  \put(-1,-2.2){\line(1,0){.5}}
     \put(-1.6,-2.2){\makebox{$\ell_{i,0}$}}

  \put(.5,-2.5){\line(1,0){.5}}
     \put(1.2,-2.5){\makebox{$\{\ell_{i, x^2_i}\}_{i \in [n] / [n_1]}$}}
     
  \put(-1,-4.5){\makebox(2,2){$\underrightarrow{ \forall i \in  [n]/[n_1] }$}}


\end{picture}
\caption{Messages in Yao's Garbeled Circuit}
\label{fig:message}
\end{center}
\end{figure}






%\paragraph{Malicious Bob}
%Alice semi-honest, and oblivious transfer is maliciously secure.
%Holds against malicious $Bob^*$
% What of deliberate circuit that shows first input  

\subsection{Construction of Garbled Circuits}

We would like to garble a circuit such that there are two keys for each input wire.
Correctness should be that 
given one of the two keys for each wire we can compute the output for the inputs those keys correspond to.
Security should be that 
given one key for each wire you can only learn the output, not the actual inputs.

%---

We build the circuit as a bunch of NAND gates that outputs one bit. 
If more bits are required, this can be done multiple times.
NAND gates can create any logic needed. 
We define the following sets:
\begin{align*}
W &= \text{the set of wires in the circuit}\\
G &= \text{the set of gates in the circuit.}
\end{align*}

For  each wire in the circuit, sample two keys
to label the possible inputs $0$ and $1$  to the wire
\[
\forall w \in W  \quad S_w^0, S_w^1 \,  \leftarrow{} \{0,1\}^k.
\]
We can think of these as the secret keys to an encryption scheme
(Gen, Enc, Dec).
For such a scheme we can always replace the secret key with the random bits fed into Gen.


\paragraph{Wires}
For each wire in the circuit we will have an invariant that the evaluator can only get one of the wires two encrypted values.
Consider an internal wire fed by the evaluation of a gate. The gate receives two encrypted values as inputs
and produces one encrypted output. The output will be one of the two labels for that wire and the evaluator will have no 
way of obtaining the other label for that wire. 
For example on wire $w_i$, the evaluator will only learn the value for $1$,  $S_{w_i}^1$.
We ensure this for the input wires by giving the evaluator only one of the two encrypted values for the wire.

\paragraph{Gates}
For every gate in the circuit we create four cipher texts. 
For each choice of inputs we encrypt the output key under each of the input keys. 
Let gate $g$ have inputs $w_1, w_2$ and output $w_3$,
\begin{align*}
e_g^{00} &= \Enc_{S_{w_1}^0} ( \Enc_{S_{w_2}^0}  ( S_{w_3}^1, 0^k) )\\
e_g^{01} &= \Enc_{S_{w_1}^0} ( \Enc_{S_{w_2}^1}  ( S_{w_3}^1, 0^k) )\\
e_g^{10} &= \Enc_{S_{w_1}^1} ( \Enc_{S_{w_2}^0}  ( S_{w_3}^1, 0^k) )\\
e_g^{11} &= \Enc_{S_{w_1}^1} ( \Enc_{S_{w_2}^1}  ( S_{w_3}^0, 0^k) ).
\end{align*}
We add $k$ zeros at the end.

\paragraph{Final Output}
For the final output wire, $S_{out}$, we can just give out their values,
\begin{align*}
S_{out}^0 &\text{ corresponds to 0}\\
S_{out}^1 &\text{ corresponds to 1.}
\end{align*}

\paragraph{$\bold{\tilde{C}}$}
For each gate, Alice sends Bob a random permutation of the set of four encrypted output values.
\[
\{e_g^{C_1, C_2} \} \quad \forall g \in G \quad C_1, C_2 \in \{0,1\}.
\]
For each gate, Alice sends Bob a random permutation of the set of four encrypted output values

\paragraph{Evaluation}
With an encrypted gate $g$,
input keys $S_{w_1} \, S_{w_2}$ for the input wires,
and four randomly permuted encryptions of the output keys, $e_g^{a}, e_g^{b}, e_g^{c}, e_g^{d}$,
Bob can evaluate the gate to find the corresponding key $S_{w3}$ for the output wire.
Bob can decrypt each of the encrypted output keys until he finds one that decrypts 
to a string ending in the proper number of $0$'s, which is very likely to contain the proper output key.
We can increase the probability of the correct key by increasing the number of $0$'s. 
\[
\exists  i \in \{a, b, c, d\} : \Dec_{S_{w_2}} ( \Dec_{S_{w_1}} ( e_g^{i} ))  = S_{w_3}, 0^k
\]

Given input wire labels 
$\{ \ell_{i, x_i} \}_{i \in [n]}$
the complete encrypted circuit $\tilde{C}$ is evaluated by working up from the input gates. 

%$l_{i,b} = \{S_{i,b}\}$

%as with PRF encryption scheme
%$Enc(_s(m) = (r,  m \oplus F_s(r)$


The evaluator should not be able to infer anything except what they could infer in the ideal world.
As a simple example, if the evaluator supplies one input to a circuit of just one NAND gate,
 they would be able to infer the input of the other party. However, this is true is the ideal world as well.

\section{Proof Intuition}

What intuition can we offer that the 
distribution of $\tilde{C}$ with one label per input wire 
is indistinguishable from what which a simulator could produce with access to the output?
%
For each input wire we are only given one key.
As we are doing double encryption,
for each input gate we only have the keys needed to decrypt one of the four possible outputs.
The other three are protected by semantic security.
%
So from each input gate we learn only one key compounding to its output wire.
As the output labels were randomized, we also do not know if that key corresponds to a 0 or a 1. 
%
For the next level of gates we again have only one key per input wire, and our argument continues. 
%
 So for each wire of the circuit we can only know one key corresponding to an output value for the wire. 
 Everything else is random garbage.
% 
As we control the mapping from output keys to output values, we can set this to whatever is needed to
match the expected output. 


Security only holds for evaluation of the circuit with one set of input values and 
we assume that the circuit is combinatorial and thus acyclic. 

% with two input all 0 or all 1 all broken
%  even with just 2 keys for one  input wire - broken. 




% !TEX root = collection.tex

\section{Malicious attacker intead of semi-honest attacker}

The assumption we had before consisted of a semi-honest attacker instead of a malicious attacker. A malicious attacker does not have to follow the protocol, and may instead alter the original protocol. The main idea here is that we can convert a protocol aimed at semi-honest attackers into one that will work with malicious attackers.

At the beginning of the protocol, we have each party commit to its inputs:
Given a commitment protocol $com$, Party 1 produces
\begin{center}
$c_1 = com(x_1; w_1)$ \\
$d_1 = com(r_1; \phi_1)$ \\
\end{center}
Party 2 produces
\begin{center}
$c_2 = com(x_2; w_2)$\\
$d_2 = com(r_2; \phi_2)$
\end{center}

We have the following guarantee: $\exists x_i, r_i, w_i, \phi_i$ such that $c_i = com(x_i; w_i) \wedge d_i = com(r_i; \phi_i) \wedge t = \pi(i,\text{transcript}, x_i, r_i)$, where transcript is the set of messages sent in the protocol so far.

Here we have a potential problem. Since both parties are choosing their own random coins, we have to be able to enforce that the coins are \emph{indeed} random. We can solve this by using the following protocol:

\begin{center}
  \begin{picture}(200,100)(10,20)
    \put(20, 90){$d_1 = com(s_1; \phi_1)$}
    \put(20,80){\vector(1,0){50}}
    \put(150, 90){$d_2 = com(s_2; \phi_2)$}
    \put(200, 80){\vector(-1,0){50}}

    \put(20, 60){$s_2^{'}$}
    \put(20,50){\vector(1,0){50}}
    \put(200, 60){$s_1^{'}$}
    \put(200, 50){\vector(-1,0){50}}
  \end{picture}
\end{center}

We calculate $r_1 = s_1 \oplus s_1^{'}$, and $r_2 = s_2 \oplus s_2^{'}$. As long as one party is picking the random coins honestly, both parties would have truly random coins.

Furthermore, during the first commitment phase, we want to make sure that the committing party actually knows the value that is being committed to. Thus, we also attach along with the commitment a zero-knowledge proof of knowledge (ZK-PoK) to prove that the committing party knows the value that is being committed to.

\subsection{Zero-knowledge proof of knowledge (ZK-PoK)}

\begin{definition}[ZK-PoK] Zero-knowlwedge proof of knowledge (ZK-PoK) is a zero-knowledge proof system $(P,V)$ with the property proof of knowledge with knowledge error $\kappa$:

$\exists$ a PPT $E$ (knowledge extractor) such that $\forall x \in L$ and $\forall P^{*}$ (possibly unbounded), it holds that if $\Pr[Out_V(P^{*}(x,w) \leftrightarrow V(x))]> \kappa(x)$, then 
\[ \Pr[E^{P^*}(x) \in R(x)] \geq \Pr[Out_V(P^{*} \leftrightarrow V(x))] = 1]- \kappa(x).\]
Here we have $L$ be the language, $R$ be the relation, and $R(x)$ is the set such that $\forall w \in R(x)$, $(x, w) \in R$.
\end{definition}

Given a zero-knowledge proof system, we can construct a ZK-PoK system for statement $x\in L$ with witness $w$ as follows:
\begin{center}
  \begin{picture}(300,300)(10,20)

    \put(10, 290){$P$}
    \put(290, 290){$V$}

    \put(10, 270){$r \leftarrow \{0, 1\}^{|w|}$}

    \put(100, 260){$c_1 = com(r; \omega)$}
    \put(100, 250){$c_2 = com(r \oplus w; \phi)$}
    \put(100, 240){\vector(1,0){100}}

    \put(150, 210){$b$}
    \put(200, 200){\vector(-1,0){100}}

    \put(120, 160){if $b = 0$, open $c_1$ to reveal $r$}
    \put(120, 150){else open $c_2$ to reveal $r \oplus w$}
    \put(100, 140){\vector(1,0){100}}

    \put(120, 60){\framebox(50,50)[c]{ZK Proof}}
  \end{picture}
\end{center}

The last ZK proof proves that $\exists r, w, \omega, \phi$ such that $(x, w) \in R$ and $c_1 = com(r; \omega)$, $c_2 = com(r \oplus w; \phi)$.


\section*{Exercises}
\begin{exercise}
Given a (secure against malicious adversaries) two-party secure computation protocol (and nothing else) construct a (secure against malicious adversaries) three-party secure computation protocol.
\end{exercise}

\input{lec27-F24}
\input{lec28-F24}

% The back matter contains appendices, bibliographies, indices, glossaries, etc.



\backmatter

\bibliography{cryptobib/abbrev0,cryptobib/crypto}
%\bibliographystyle{plainnat}
\bibliographystyle{plain}


%\printindex

\end{document}

